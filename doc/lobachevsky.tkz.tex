\begin{figure}[!thb]
  \centering
  \begin{tikzpicture}[scale=0.7]
    \coordinate [label=below left:$P$] (P) at (0, 0);
    \coordinate [label=left:$Q$] (Q) at (0, 1.2cm);
    \coordinate [label=below:$R$] (R) at (6cm, 0) {};
    \draw (P) -- ($ (R) + .5*(2+rand, 0) $);
    
    \node at ($ (P)!.5!(Q) $) [left] {$a$};
    \node at ($ (Q) + (-50:0.5cm) $){$\theta$};

    \begin{scope}
      \draw[clip] (P) -- node[left]{$a$} (Q) .. controls ($ (P)!.5!(R) $)
      and ($ (P)!.9!(R) $) .. (R);
      \draw (Q) circle(0.3cm);
    \end{scope}
    \foreach \p in {P, Q, R} \draw node[dot] at (\p) {};
  \end{tikzpicture}
  \caption{Предположение Лобачевского: $\theta = F(a)$}
  \label{fig:lobachevsky}
\end{figure}
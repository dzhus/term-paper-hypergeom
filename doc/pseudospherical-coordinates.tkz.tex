\begin{figure}[thb]
  \centering
  \begin{tikzpicture}[samples=50,smooth,raw gnuplot]
    \coordinate[label=below:$O$] (O) at (0,0) {};
    \coordinate (P) at (3.2,1.7) {};
    \coordinate (Pp) at ($ (P)+(-4.3,0) $);

    % axis
    \draw[->] (O) -- ++(5,0) node[label=below:$x$] {};
    \draw[->] (O) -- ++(0,3) node[label=left:$z$] {};
    \draw[->] (O) -- ++(-2,-1.5) node[label=above:$y$] {};

    % phi
    \begin{scope}
      \clip (Pp)--(O)--++(-2,-1.5);
      \draw ($ (O)+(0.05,0) $) circle(0.3);
    \end{scope}
    \node at ($ (O)+(-0.5,0) $) {$\phi$};

    % chi
    \begin{scope}
      \clip (O)--(P)--(3,0);
      \draw[double distance=1pt] (O) circle(0.7);
    \end{scope}
    \node at ($ (O)+(0.90,0.20) $) {$\chi$};
    
    % radius and its projection
    \draw[thick] (O) -- node[pos=0.7,label=above:$\rho$] {}
    (P);
    \draw[densely dotted] (P) -- (Pp);
    \draw[densely dashed] (O)  -- (Pp);
    
    \node[dot] at (O) {};
    \node[dot] at (P) {};
  \end{tikzpicture}
  \caption{Псевдосферические координаты $(\rho, \chi, \phi)$}
  \label{fig:pseudospheric-coords}
\end{figure}
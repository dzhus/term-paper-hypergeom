\begin{figure}[thb]
  \centering
  \begin{tikzpicture}
    % surface
    \draw (0,-1) to[out=25,in=150] (4,-1.5)
                 to[out=90,in=215] (6, 2.5)
                 to[out=140,in=20] (2, 3)
                 to[out=215,in=90] (0,-1);


    % tangent point
    \coordinate [] (M) at (2.7,1) {};
    
    % curve and its projection of onto plane (hatched path)
    \draw[pattern=vertical lines] (1,-.5) node[right] {$\gamma$} to[out=75,in=190] (M)
                  to[out=5,in=220] (4.7,1.5)
                  -- (4.7,2)
                  to[out=240,in=10] (M)
                  to[out=190,in=75] node[above] {$\gamma_\pi$}
                  (1,.2) -- cycle;
         
    % tangent plane
    \draw[fill=white,fill opacity=.5] (-.5,-.2)
                                    --(5,-.2)
                                    --(6,2.2)
                                    --(0.5,2.2)
                                    --cycle;
    \draw[pattern=north east lines,pattern color=gray!60,
          opacity=.5] (-.5,-.2)
                      --(5,-.2)
                      --(6,2.2)
                      --(0.5,2.2)
                      --cycle;

    % surface and plane labels                      
    \node[above left] at (5, -.2) {$\pi$};                      
    \node[below right] at (2, 3) {$\Phi$};

    % projection (again)
    \draw (4.7,2) to[out=240,in=10] (M)
                  to[out=190,in=75] node[above] {$\gamma_\pi$}
                  (1,.2);
                  
    \foreach \p in {M} \draw node[dot,label=below:$\p$] at (\p) {};
  \end{tikzpicture}
  \caption{Кривая $\gamma$ на поверхности $\Phi$ и её проекция
    $\gamma_\pi$ на касательную в точке $M$ плоскость $\pi$}
  \label{fig:geodesic-curvature}
\end{figure}
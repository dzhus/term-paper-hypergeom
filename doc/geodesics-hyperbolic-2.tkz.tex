\begin{figure}[thb]
  \centering
  \begin{tikzpicture}[samples=50,smooth]
    \coordinate[dot,label=left:$O$] (O) at (0,0) {};
    
    \begin{scope}[raw gnuplot,thick]
      \draw plot[id=geo1] function{set polar; set parametric; plot [0.54930615:2.5] asin(cosh(t)/(2*sinh(t))),t;};
      \draw plot[id=geo2] function{set polar; set parametric; plot [0.54930615:2.5] asin(cosh(t)/(2*sinh(t)))+2*pi/6,t;};
      \draw plot[id=geo3] function{set polar; set parametric; plot [0.54930615:2.5] asin(cosh(t)/(2*sinh(t)))+4*pi/6,t;};
      \draw plot[id=geo4] function{set polar; set parametric; plot [0.54930615:2.5] asin(cosh(t)/(2*sinh(t)))+6*pi/6,t;};
      \draw plot[id=geo5] function{set polar; set parametric; plot [0.54930615:2.5] asin(cosh(t)/(2*sinh(t)))+8*pi/6,t;};
      \draw plot[id=geo6] function{set polar; set parametric; plot [0.54930615:2.5] asin(cosh(t)/(2*sinh(t)))+10*pi/6,t;};
      \draw plot[id=geo1] function{set polar; set parametric; plot [0.54930615:2.5] pi-asin(cosh(t)/(2*sinh(t))),t;};
      \draw plot[id=geo2] function{set polar; set parametric; plot [0.54930615:2.5] pi-asin(cosh(t)/(2*sinh(t)))+2*pi/6,t;};
      \draw plot[id=geo3] function{set polar; set parametric; plot [0.54930615:2.5] pi-asin(cosh(t)/(2*sinh(t)))+4*pi/6,t;};
      \draw plot[id=geo4] function{set polar; set parametric; plot [0.54930615:2.5] pi-asin(cosh(t)/(2*sinh(t)))+6*pi/6,t;};
      \draw plot[id=geo5] function{set polar; set parametric; plot [0.54930615:2.5] pi-asin(cosh(t)/(2*sinh(t)))+8*pi/6,t;};
      \draw plot[id=geo6] function{set polar; set parametric; plot [0.54930615:2.5] pi-asin(cosh(t)/(2*sinh(t)))+10*pi/6,t;};
    \end{scope}
  \end{tikzpicture}
  \caption{Геодезические линии с разными значениями параметра $C_2$}
  \label{fig:geodesics-hyperbolic-2}
\end{figure}

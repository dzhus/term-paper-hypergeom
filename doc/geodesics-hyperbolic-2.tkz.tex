\begin{figure}[!thb]
  \centering
  \begin{tikzpicture}[axis/.style={draw=gray,thin},samples=50,smooth,scale=0.9]
    \coordinate[label=left:$O$] (O) at (0,0) {};
    
    % cartesian coordinates
    \draw [axis,->] (O)--($ (O) + (0,3) $) node[label=left:$v$] {};
    \draw [axis,->] (O)--($ (O) + (3,0) $) node[label=above:$u$] {};
    % phantom line (hack to fix bounding box)
    \draw [draw=none] (O)--($ (O) - (3,0) $);

    \begin{scope}[raw gnuplot]
      \foreach \p in {0,...,11}
               \draw plot function{set polar; set parametric; plot [0.54930615:2.5] asin(cosh(t)/(2*sinh(t)))+\p*pi/6,t;}
                     plot function{set polar; set parametric; plot [0.54930615:2.5] pi-asin(cosh(t)/(2*sinh(t)))+\p*pi/6,t;};
    \end{scope}

    \node[dot] at (O) {};
  \end{tikzpicture}
  \caption[Геодезические линии с разными значениями параметра
  $C_2$]{Геодезические линии с разными значениями параметра $C_2$ при
    фиксированном \mbox{$C_1=2$}}
  \label{fig:geodesics-hyperbolic-2}
\end{figure}

\documentclass{article}
\usepackage[utf8x]{inputenc}
\usepackage[english,russian]{babel}

\usepackage{amsmath,amssymb}

% provides enumerated cases environment
\usepackage{cases}

% provides \wideparen
\usepackage{yhmath}

% rich title
\usepackage{titling}

% Include bibliography in TOC
\usepackage[numbib,nottoc]{tocbibind}

% TikZ
\usepackage{tikz}
\usetikzlibrary{calc,through,positioning,patterns,decorations.markings}
\tikzset{dot/.style={circle,fill=black,scale=0.5}}

\numberwithin{equation}{section}

% Russian traditions
\renewcommand{\epsilon}{\varepsilon}
\renewcommand{\phi}{\varphi}
\DeclareMathOperator{\arcsh}{arcsh}
\DeclareMathOperator{\arccth}{arccth}
\usepackage{misccorr}

\providecommand{\const}{\mathrm{const}}
\newcommand{\krist}[3]{\Gamma^{\phantom{#1 #2}#3}_{#1 #2}}
\newcommand{\neword}[1]{\textbf{#1}}
\newcommand{\scalmult}[2]{{\left \langle #1 \right \rangle}_{#2}}
\renewcommand{\vec}{\mathbf}

\providecommand{\arc}[1]{\wideparen{#1}}
\providecommand{\abs}[1]{\left \lvert{#1}\right \rvert}
\begin{document}

\author{Дмитрий Джус}
\title{Курсовая работа\\ по дифференциальной геометрии\\ на тему \\
  \Huge{«Спирали на плоскости Лобачевского»}}
\pretitle{\begin{center}\LARGE}
\posttitle{\par\end{center}\vskip 3pc}
\date{}
\maketitle
\thispagestyle{empty}

\clearpage

\tableofcontents
\clearpage

\listoffigures
\clearpage

\part{Введение}

\section{Исторический очерк}
\label{sec:intro}

В целях общего ознакомления с контекстом теории рассмотрим краткую
историю развития неевклидовой геометрии, в том числе геометрии
Лобачевского. Очерк подготовлен на основе \cite{milnor82} и
\cite{norden56}.

Геометрия Лобачевского (или гиперболическая геометрия) возникла в
попытках доказать или опровергнуть пятый постулат евклидовой
геометрии:

\begin{quote}
  Через любую точку, не лежащую на данной прямой, можно провести
  прямую, и притом только одну.
\end{quote}

Возможность опровержения этого начала занимала умы математиков и
философов на протяжении многих веков. Иммануил Кант с позиций своей
теории познания признавал \emph{априорный} характер геометрии и
утверждал, что основания геометрии имеют доопытное, очевидное
происхождение и черпаются из «чистого воззрения». С такой точки
зрения, лишь пятый постулат не обладал необходимым свойством
самоочевидности, и возможность его доказательства представлялась
равносильной возможности априорного обоснования всей геометрии.

Таких мыслей изначально придерживался и Карл Гаусс. Уверенность в
возможности доказательства пятого постулата долго не покидает его, и
лишь в 1817 году он сомневается в этом, указывая на вероятную
неаприорность геометрии. В дальнейшие годы Гаусс значительно изменяет
своё видение теории параллельных, обнаруживая, что геометрия без
аксимы Евклида совершенно последовательна. При жизни Гаусс своих работ
по началам неевклидовой геометрии, однако, не публиковал.

В 1826 году Николай Иванович Лобачевский публикует своё сочинение
«О началах геометрии», содержащее непротиворечивую геометрическую
теорию, построенную не с использованием постулата Евклида о
параллельных, а в предположении о том, что «угол параллелизма»
наоборот не является прямым и вовсе зависит от длины отрезка $a$. На
рисунке \ref{fig:lobachevsky} угол $\theta < \pi/2$ прямоугольного
треугольника $PQR$ является функцией $F(a)$ отрезка постоянной длины
$a$, увеличиваясь при $R \to \infty$.

\begin{figure}[thb]
  \centering
  \begin{tikzpicture}[scale=0.7]
    % triangle vertices
    \coordinate [label=below left:$P$] (P) at (0, 0);
    \coordinate [label=left:$Q$] (Q) at (0, 1.2cm);
    \coordinate [label=below:$R$] (R) at (6cm, 0) {};

    \draw (P) -- ($ (R) + .5*(2+rand, 0) $);

    % angle labels
    \node at ($ (P)!.5!(Q) $) [left] {$a$};
    \node at ($ (Q) + (-50:0.5cm) $){$\theta$};

    % hypotenuse and angle arc
    \begin{scope}
      \draw[clip] (P) -- node[left]{$a$} (Q) .. controls ($ (P)!.5!(R) $)
      and ($ (P)!.9!(R) $) .. (R);
      \draw (Q) circle(0.3cm);
    \end{scope}

    % dot marks
    \foreach \p in {P, Q, R} \draw node[dot] at (\p) {};
  \end{tikzpicture}
  \caption{Предположение Лобачевского: $\theta = F(a)$}
  \label{fig:lobachevsky}
\end{figure}

Указывая, что на предельной поверхности будет иметь место геометрия
Евклида, Лобачевский прямо переходит к доказательству
непротиворечивости тригонометрии. Впоследствии он также строит
дифференциальную и аналитическую геометрию своего пространства,
окончательно доказывая его непротиворечивость с одной стороны и
соответствие абсолютной геометрии с другой.

Работу по неевклидовой геометрии также опубликовал Янош Больаи в конце
1820-ых годов.

Дальнейшее развитие взглядов на геометрию Лобачевского было неотрывно
связано с дифференциальной геометрией.

В 1827 году Гаусс публикует свои работы по дифференциальной геометрии,
в которых даются определения таких понятий, как криволинейные
координаты, квадратичные формы поверхности, геодезическая кривизна.
Важным результатом Гаусса явилось выделение \emph{внутренней}
геометрии поверхности и доказательство инвариантности гауссовой
кривизны поверхности относительно преобразования изгибания. Известно,
что Гаусс также знал о псевдосфере — поверхности постоянной
отрицательной кривизны.

Ученик Гаусса Фердинанд Миндинг также рассматривал поверхности
постоянной кривизны, их движения по себе, заметив, что формулы
тригонометрии геодезических треугольников на поверхностях постоянной
отрицательной кривизны могут быть получены из тригонометрии на сфере
путём замены действительного коэффициента $\sqrt{k}$ на мнимый.

Опираясь на результаты Миндинга, Эудженио Бельтрами в 1868 публикует
своё сочинение об интерпретации неевклидовой геометрии. Согласно ему,
геометрия Лобачевского может быть осуществлена в евклидовом
пространстве как внутренняя геометрия поверхностей постоянной
отрицательной кривизны. Этот результат был воспринят как
доказательство непротиворечивости неевклидовой геометрии, хотя
Гильберт впоследствии указал на невозможность полной реализации
геометрии Лобачевского на таких поверхностях.

С этими открытиями и переводом работ Лобачевского повысился всеобщий
интерес к проблемам основания геометрии. Бернгард Риман рассмотрел
пространство $n$ измерений как многообразие элементов, в котором
каждый задаётся набором из $n$ независимых переменных, а геометрию
этого пространства можно определить квадратичной формой дифференциалов
независимых переменных (линейным элементом), которая задаёт квадрат
расстояния между элементами многообразия, что позволяет далее вводить
понятия длины, кратчайших или геодезических линий, угла, объёма. Риман
указал, что пространство, линейный элемент которого приводится к сумме
квадратов дифференциалов, является многомерным аналогом евклидового, а
также обобщил это понятие на пространства, допускающих движения по
себе с таким же числом степеней свободы, как и евклидово. Характер
этих пространств заключён в том, что их кривизна не зависит от
направления площадки.

Лобачевский положил в основу своей геометрии формулы тригонометрии, а
Риман выполнил замысел Лобачевского ещё более законно, показав
возможность построения геометрии на чисто аналитической основе.
Впоследствии Бельтрами была обнаружена ещё более глубокая связь между
геометрией Лобачевского и системой Римана: трёхмерное пространство
Римана постоянной отрицательной кривизны совпадает с пространством
Лобачевского. Так непротиворечивость неевклидовой геометрии была
доказана строго.

Одновременно развивалась и проективная геометрия. Важной с точки
зрения обобщения геометрических понятий явилась мысль Юлиуса Плюккера
о том, что в качестве элементов геометрии могут быть рассмотрены не
только точки. В то же время, Феликс Клейн, опираясь на работы Артура
Кэли по проективной метрике, дал новую интерпретацию геометрии
Лобачевского с точки зрения геометрии проективной. Модель Клейна в
круге, приведённая на рисунке \ref{fig:klein-model}, является
полноценной моделью пространства Лобачевского, в которой метрика
задана следующим образом (см. \cite{prasolov04}, \cite{zaslavsky04}):

\begin{equation}\label{eq:projective-metric}
  \rho(A, B) = \frac{R}{2} \abs{\ln{\left (
        \frac{X-A}{X-B}:\frac{Y-A}{Y-B} \right )}}
\end{equation}

В модели Клейна точки изображаются точками, прямые — хордами.
Бесконечное множество прямых, проходящих через точку $P$ параллельно
данной прямой $l$, заполняют часть круга между хордами $XM$ и $YN$.

\begin{figure}[thb]
  \centering
  \begin{tikzpicture}[]
    \coordinate [label=below left:$O$] (O) at (0, 0) {};
    \coordinate [label=above right:$X$] (X) at (60:3) {};
    \coordinate [label=below left:$Y$] (Y) at (200:3) {};
    \coordinate [label=above:$A$] (A) at ($ (Y)!.2!(X)$) {};
    \coordinate [label=above:$B$] (B) at ($ (X)!.3!(Y)$) {};
    \coordinate [label=above left:$P$] (P) at (285:1.5) {};

    % points somewhere outside the disk
    \coordinate (m) at ($ (X)!3!(P) $);
    \coordinate (n) at ($ (Y)!3!(P) $);

    % disk model
    \node [draw,thick,circle through=(X)] (circle) at (O) {};

    % chords
    \draw (X) --node[auto]{$l$} (Y);
    \coordinate [label=below:$M$] (M) at (intersection 1 of X--m and circle) {};
    \coordinate [label=right:$N$] (N) at (intersection 1 of Y--n and circle) {};
    \draw (X) -- (M);
    \draw (Y) -- (N);

    % a bundle of parallel lines
    \begin{scope}
      \clip (O) circle(3);
      \foreach \a in {5,15,25,35,45,55,65} \draw[ultra thin] ($ (P)+(180+\a:5) $) -- ($ (P)+(\a:5) $);
    \end{scope}

    %dot marks
    \foreach \p in {O, X, Y, A, B, P, M, N} \draw node[dot] at (\p) {};
  \end{tikzpicture}
  \caption{Модель Клейна в круге}
  \label{fig:klein-model}
\end{figure}

В конце XIX века Софус Ли создаёт теорию непрерывных групп
преобразований с многочисленными приложениями её к дифференциальным
уравнениями и геометрии.

Понятие группы преобразований стало одним из центральных в
«Эрлангенской программе» Клейна, который в ней дал обобщённоё
понимание геометрии, определив её задачу следующим образом:

\begin{quote}
  Дано многообразие и в нём группа преобразований; нужно исследовать
  те свойства образов, принадлежащий многообразию, которые не
  изменяются от преобразований группы.
\end{quote}

Из такого определения напрямую вытекает существование различных
геометрий, которые различаются характером элементов и, что наиболее
важно, строением своей группы. Например, группа круговых
преобразований плоскости изоморфна группе движений пространства
Лобачевского, а значит, круговая геометрия плоскости даёт ещё одну
интерпретацию геометрии пространства Лобачевского.

Построение неевклидовой геометрии, впервые выполненное Лобачевским,
привело к интенсивному развитию геометрической теории, продолжающееся
и до сих пор по пути углубления и расширения сочетания групповой и
дифференциально-геометрической или более общей топологической точки
зрения.

\clearpage
\part{Практическая часть}

\section{Предварительные данные}
Рассматривается плоскость Лобачевского, первая квадратичная форма
задана в полярных координатах $(\rho, \phi)$ следующим образом:

\begin{equation}\label{eq:lin-elt}
  {ds}^2 = d\rho^2+\sh^2{\rho}\,d\phi^2
\end{equation}

\subsection{Линейный элемент}

Первая квадратичная форма (называемая линейным элементом) даёт
выражение для дифференциала дуги кривой.

В данном случае (форма невырожденная, положительно определённая на
всех точках) можно также говорить о том, что задана
\neword{риманова метрика} (см. \cite{dubrovin98}).

Известно, что линейный элемент выражается как
\mbox{$ds^2=g_{ij}dx^idx^j$}, где коэффициенты $g_{ij}$ образуют
функциональную матрицу $G(x)$, соответствующую рассматриваемой
римановой метрике. В данном случае она имеет вид

\begin{equation*}
  \begin{pmatrix}
    1 && 0 \\
    0 && \sh^2{\rho}
  \end{pmatrix}
\end{equation*}

При заданной первой квадратичной форме скалярное произведение
$\scalmult{\vec{x}, \vec{y}}{}$ двух векторов $\vec{x}=(x^1,x^2)$ и
$\vec{y}=(y^1,y^2)$ может быть вычислено по формуле
\begin{equation}\label{eq:scalmult}
  \scalmult{\vec{x}, \vec{y}}{} = g_{ij}x^i y^j
\end{equation}

Задание линейного элемента на некоторой системе криволинейных
координат $(x^1, \dotsc , x^n)$ даёт возможность исследовать
\emph{внутреннюю} геометрию поверхности, в том числе, вычислять длины
дуг произвольных кривых:

\begin{equation}\label{eq:riemann-curve-length}
  l(\gamma)^a_b = \int \limits^a_b {\sqrt{\sum_{i,j}{g_{ij}
        \frac{dx^i}{dt} \frac{dx^j}{dt}}} dt} = \int \limits^a_b ds
\end{equation}

\subsection{Природа пространства}
\label{sec:what-is-up}
Рассмотрим происхождение рассматриваемого пространства. Материал
данного раздела подготовлен на основе \cite{fomenko00} и
\cite{dubrovin98}.

По аналогии с положительно определёнными римановыми метриками могут
быть введены \neword{индефинитные метрики}, соответствующая первая
квадратичная форма которых не обязательно обладает свойством
положительной определённости.

Примером индефинитных метрик служат \neword{псевдоевклидовы метрики}.
Для построения псевдоевклидовой метрики достаточно рассмотреть обычное
евклидово пространство $\mathbb{R}^n$ и задать в каждой его точке
билинейную форму с постоянными коэффициентами вида
\begin{equation*}
  \scalmult{\vec{\xi}, \vec{\eta}}{s} = -\sum_{i=1}^s {\xi^i \eta^i} +
\sum_{j=s+1}^n {\xi^j \eta^j}
\end{equation*}

Соответственно, длина дуги в этом \neword{псевдоевклидовом
  пространстве} $\mathbb{R}^n_s$ выражается по формуле
\begin{equation*}
  l(\gamma)^a_b = \int \limits^a_b {\sqrt{-\sum_{i=1}^s
      {\left (\frac{dx^i}{dt} \right )}^2 + \sum_{j=s+1}^n
      {\left (\frac{dx^j}{dt} \right )}^2}dt}
\end{equation*}

В $\mathbb{R}^n_s$ длина вектора может быть не только действительной,
но и нулевой или комплексной.

С введением возможности измерять расстояния в $\mathbb{R}^n_s$ может
быть рассмотрена \neword{псевдосфера} $S^{n-1}$ как множество точек,
равноудалённых от начала координат, при этом радиус не обязательно
является действительным.

Так, псевдосфера нулевого радиуса описывается уравнением второго порядка
\begin{equation*}
  -\sum_{i=1}^s\left (x^i \right)^2 + \sum_{i=s+1}^n\left(x^j \right)^2 = 0  
\end{equation*}
где $x^1, \dotsc, x^n $ — декартовы координаты в $\mathbb{R}^n$, в
котором мы моделируем псевдоевклидово пространство $\mathbb{R}^n_s$.

Рассмотрим моделируемое в $\mathbb{R}^3$ пространство
$\mathbb{R}^3_1$; при помощи $x, y, z$ будем обозначать обычные
декартовы координаты в $\mathbb{R}^3$, тогда
\mbox{$\scalmult{\vec{\xi}, \vec{\xi}}{1} = -x^2 + y^2 + z^2$}, а
квадрат дифференциала длины дуги в этом пространстве выражается как
\begin{equation}\label{eq:R^3_1-linear-element}
  ds^2 = -dx^2 + dy^2 + dz^2
\end{equation}

Рассмотрим псевдосферу мнимого радиуса в $\mathbb{R}^3_1$. Это —
двуполостный гиперболоид, задаваемый уравнением
\begin{equation}\label{eq:pseudosphere}
  -\alpha^2 = -x^2 +y^2 + z^2
\end{equation}
здесь $\alpha \in \mathbb{R}$.

Изучим некоторые свойства его внутренней геометрии, сначала
спроецировав точки гиперболоида на плоскость следующим образом.

Будем считать центром псевдосферы $S^2_1=\left \{-\alpha^2 = -x^2 + y^2 +
  z^2\right \}$ точку $O=(0, 0, 0)$, а северным и южным полюсами — точки
$N=(-\alpha, 0, 0)$ и $S=(\alpha, 0, 0)$ соответственно.

\begin{figure}[thb]
  \centering
  \begin{tikzpicture}[scale=1,axis/.style={color=gray,thin,->}]
    \coordinate [label=below left:$O$] (O) at (0, 0);
    \coordinate [label=right:$z$] (z) at (0,4);
    \coordinate [label=below:$x$] (x) at (4,0);

    % poles
    \coordinate [label=below left:$N$] (N) at ($ (O)!-.5!(x) $);
    \coordinate [label=below right:$S$] (S) at ($ (O)!.5!(x) $);
    
    % axis
    \draw [axis] ($ (O)!-1!(x) $) -- (O) -- (x);
    \draw [axis] ($ (O)!-1!(z) $) -- (O) -- (z);

    % projection disk
    \coordinate (M) at ($ (O)!.5!(z) $) {};
    \coordinate [label=right:$D^2$] (L) at ($ (O)!-.5!(z) $) {};
    \draw [ultra thick] (L)--(M);

    % hyperbola
    % right branch
    \path[draw] (S) plot[domain=-1:1,very thick]
                    ({(exp(\x)+exp(-\x))}, {(exp(\x)-(exp(-\x)))});
    % left branch
    \path[draw] (N) plot[domain=-1:1,very thick]
                    ({(-exp(\x)-exp(-\x))}, {(-exp(\x)+(exp(-\x)))});
    % a point on hyperbola
    \coordinate[label=above left:$P$] (P) at ({(exp(.7)+exp(-.7))}, {(exp(.7)-(exp(-.7)))});

    % projection
    \draw (P) -- (N);
    \coordinate [label=above left:$\psi(P)$] (Pp)
                at (intersection of P--N and M--L) {};

    % dot marks
    \foreach \p in {O,N,S,P,Pp} \node[dot] at (\p){};
  \end{tikzpicture}
  \caption[Стереографическая проекция псевдосферы]{Стереографическая
    проекция псевдосферы $S^2_1$ мнимого радиуса, смоделированной в
    $\mathbb{R}^3$, на плоскость $YOZ$ (сечение $XOZ$)}
  \label{fig:pseudosphere-projection}
\end{figure}

Не ограничивая общности, для простоты рассмотрим ту часть псевдосферы,
которая определена неравенством $x>0$.

Выберем плоскость $YOZ$, проходящую через центр псевдосферы, в
качестве плоскости проецирования. Произвольную точку $P$ гиперболоида
соединим с северным полюсом $N$. Тогда образом точки $P$ при
стереографической проекции $\psi \colon S^2_1 \to \mathbb{R}^2$ будет
точка пересечения отрезка $PN$ с плоскостью $YOZ$
(рисунок \ref{fig:pseudosphere-projection}). Видно, что образом правой
полости гиперболоида на плоскости будет внутренность диска $D^2$,
ограниченная неравенством $y^2 + z^2 < \alpha^2$, тогда как левая
полость проецируется во внешность этого диска, при этом граница диска
не принадлежит образу проекции.

Если точке гиперболоида $P = (x, y, z)$ при отображении $\psi \colon
{}_+S^2_1 \to D^2$ соответствует точка $\psi(P) = (u^1, u^2)$ на
плоскости проецирования, то их координаты связаны следующим образом
(см. \cite{fomenko00}):
\begin{equation}\label{eq:hyperboloid-to-plane}
  x = \alpha\frac{\abs{\vec{u}}^2 + \alpha^2}{\alpha^2 -
    \abs{\vec{u}}^2}, \qquad
  y = \frac{2 \alpha^2 u^1}{\alpha^2 - \abs{\vec{u}}^2}, \qquad
  z = \frac{2 \alpha^2 u^2}{\alpha^2 - \abs{\vec{u}}^2}
\end{equation}

Можно ввести в качестве точек внутренней геометрии рассматриваемой
псевдосферы пары диаметрально противоположных точек $P$ и $-P$, а
прямыми объявить линии пересечения гиперболоида с всевозможными
плоскостями $ax+by+cz=0$, проходящими через центр псевдосферы $O$.
Заметим, что параллельных прямых в данной геометрии в привычном для
нас понимании не существуют: прямые либо пересекаются в точке, либо
совпадают. Полученная геометрия называется \neword{эллиптической}.

При отображении $\psi$ прямые введённой геометрии на гиперболоиде
перейдут в дуги окружностей, пересекающие под прямым углом окружность
$y^2 + z^2 = \alpha^2$, что указывает на то, что геометрия, введённая
на псевдосфере в $\mathbb{R}^3_1$, после замены координат совпадает с
геометрией, возникающей в круге радиуса $\alpha$, взятого на
евклидовой плоскости $\mathbb{R}^2$, если в качестве точек этой
геометрии взять точки круга, а прямыми назвать дуги окружностей,
пересекающий границу круга под прямым углом. Полученная геометрия и
является \neword{геометрией Лобачевского}, а данная её модель
называется \neword{моделью Пуанкаре в круге}. В ней выполнены все
постулаты Евклида, кроме пятого. В смысле возможности проведения
параллельных прямых эта геометрия противоположна эллиптической — через
каждую точку в модели Пуанкаре проходит бесчисленное число прямых,
параллельных данной. На рисунке \ref{fig:poincare-disk-model}
изображена прямая линия $l$ и пара других параллельных ей.

\begin{figure}[thb]
  \centering
  \begin{tikzpicture}[scale=2]
    \coordinate [] (O) at (0, 0) {};
    \coordinate [label=above right:$X$] (X) at (55:1) {};
    \coordinate [label=below left:$Y$] (Y) at (200:1) {};
    \coordinate (M) at (330:1);
    \coordinate (N) at (290:1);

    % disk model
    \node [draw,thick,circle through=(X)] (circle) at (O) {};

    % lines
    \draw (X) to[out=235,in=20] node[pos=0.5,label=above:$l$]{} (Y);
    \draw (Y) to[out=20,in=150] (M);
    \draw (X) to[out=235,in=110] (N);

    
    % a bundle of parallel lines
   
    %dot marks
    \foreach \p in {X, Y} \draw node[dot] at (\p) {};
  \end{tikzpicture}
  \caption{Модель Пуанкаре в круге}
  \label{fig:poincare-disk-model}
\end{figure}

Теперь выполним следующее преобразование координат в $\mathbb{R}^3_1$.
Введём в плоскости $YOZ$ полярные координаты $(\rho, \phi)$, где
$\rho$ — угол с полярной осью $y$, и по аналогии со сферическими
координатами введём параметр $\chi$, равный углу между радиус-вектором
точки и осью $x$ (рисунок \ref{fig:pseudospheric-coords}).

\begin{figure}[thb]
  \centering
  \begin{tikzpicture}[samples=50,smooth,raw gnuplot]
    \coordinate[label=below:$O$] (O) at (0,0) {};
    \coordinate (P) at (3.2,1.7) {};
    \coordinate (Pp) at ($ (P)+(-4.3,0) $);

    % axis
    \draw[->] (O) -- ++(5,0) node[label=below:$x$] {};
    \draw[->] (O) -- ++(0,3) node[label=left:$z$] {};
    \draw[->] (O) -- ++(-2,-1.5) node[label=above:$y$] {};

    % phi
    \begin{scope}
      \clip (Pp)--(O)--++(-2,-1.5);
      \draw ($ (O)+(0.05,0) $) circle(0.3);
    \end{scope}
    \node at ($ (O)+(-0.5,0) $) {$\phi$};

    % chi
    \begin{scope}
      \clip (O)--(P)--(3,0);
      \draw[double distance=1pt] (O) circle(0.7);
    \end{scope}
    \node at ($ (O)+(0.90,0.20) $) {$\chi$};
    
    % radius and its projection
    \draw[thick] (O) -- node[pos=0.7,label=above:$\rho$] {}
    (P);
    \draw[densely dotted] (P) -- (Pp);
    \draw[densely dashed] (O)  -- (Pp);
    
    \node[dot] at (O) {};
    \node[dot] at (P) {};
  \end{tikzpicture}
  \caption{Псевдосферические координаты $(\rho, \chi, \phi)$}
  \label{fig:pseudospheric-coords}
\end{figure}

После осуществления замены переменных (перехода к псевдосферическим
координатам)
\begin{equation*}\label{eq:pseudospheric-coords}
  y = \alpha \sh{\chi} \cos{\phi}, \qquad
  z = \alpha \sh{\chi} \sin{\phi}, \qquad
  x = \alpha \ch{\chi}
\end{equation*}
уравнение псевдосферы \eqref{eq:pseudosphere} запишется в виде $\alpha
= \text{const}$. Можно получить вид \emph{римановой метрики на
  псевдосфере} в координатах $(u^1, u^2)$ модели Пуанкаре, подставляя
формулы \eqref{eq:hyperboloid-to-plane} в
\eqref{eq:R^3_1-linear-element}:
\begin{equation*}
  -(dx(u^1, u^2))^2 + (dy(u^1, u^2))^2 + (dz(u^1, u^2))^2 =
  4\alpha^4\frac{(du^1)^2 + (du^2)^2}
                {\left (\alpha^2 - (u^1)^2-(u^2)^2 \right )^2}
\end{equation*}

Если в модели Пуанкаре при $\alpha=1$ теперь ввести полярные
координаты $(\rho, \phi)$, линейный элемент представится в виде
\begin{equation}\label{eq:poincare-lin-elt-polar}
  ds^2 = 4\frac{d\rho^2 + \rho^2\,d\phi^2}{\left (1 - \rho^2 \right )^2}
\end{equation}

Наконец, при записи в псевдосферических координатах $(\rho, \chi, \phi)$
(что достигается выполнением преобразованием координат $\rho =
\cth(\chi/2)$, $\phi=\phi$) метрика приобретёт вид
\begin{equation}\label{eq:S^2_1-linear-element}
  ds^2 = d\chi^2 + \sh^2{\chi}\,d\phi^2
\end{equation}

Метрика \eqref{eq:S^2_1-linear-element} называется \neword{метрикой
  Лобачевского} в записи в псевдосферических координатах.

Таким образом, на псевдосфере $S^2_1$ мнимого радиуса объемлющим
псевдоевклидовым пространством $\mathbb{R}^3_1$ посредством
стереографического отображения $\psi$ \neword{индуцируется} риманова
метрика \eqref{eq:S^2_1-linear-element}. Отметим, что объемлющее
пространство обладает, наоборот, индефинитной метрикой
\eqref{eq:R^3_1-linear-element}.

Заметим, наконец, что выражение \eqref{eq:S^2_1-linear-element} для
линейного элемента \emph{совпадает} с предложенным к рассмотрению
\eqref{eq:lin-elt} с точностью до переобозначения координат $\chi =
\rho$.

Сопоставляя теперь $S^2_1$ с координатами $(\chi, \phi)$ и плоскость с
полярными координатами $(\rho, \phi)$, где $\rho=\chi$, получим
\neword{плоскость Лобачевского} с полярными координатами $(\rho,
\phi)$ и метрикой \eqref{eq:lin-elt}.

\subsection{Геодезическая кривизна}

\neword{Геодезической кривизной} кривой $\gamma$ на поверхности $\Phi$
в некоторой точке $P$ называют кривизна ортогональной проекции
$\gamma$ на касательную плоскость к $\Phi$, проведённую в точке
$\gamma$ (рисунок \ref{fig:geodesic-curvature}).

\begin{figure}[!thb]
  \centering
  \begin{tikzpicture}
    % surface
    \draw (0,-1) to[out=25,in=150] (4,-1.5)
                 to[out=90,in=215] (6, 2.5)
                 to[out=140,in=20] (2, 3)
                 to[out=215,in=90] (0,-1);


    % tangent point
    \coordinate [] (M) at (2.7,1) {};
    
    % curve and its projection of onto plane (hatched path)
    \draw[pattern=vertical lines] (1,-.5) node[right] {$\gamma$} to[out=75,in=190] (M)
                  to[out=5,in=220] (4.7,1.5)
                  -- (4.7,2)
                  to[out=240,in=10] (M)
                  to[out=190,in=75] node[above] {$\gamma_\pi$}
                  (1,.2) -- cycle;
         
    % tangent plane
    \draw[fill=white,fill opacity=.5] (-.5,-.2)
                                    --(5,-.2)
                                    --(6,2.2)
                                    --(0.5,2.2)
                                    --cycle;
    \draw[pattern=north east lines,pattern color=gray!60,
          opacity=.5] (-.5,-.2)
                      --(5,-.2)
                      --(6,2.2)
                      --(0.5,2.2)
                      --cycle;

    % surface and plane labels                      
    \node[above left] at (5, -.2) {$\pi$};                      
    \node[below right] at (2, 3) {$\Phi$};

    % projection (again)
    \draw (4.7,2) to[out=240,in=10] (M)
                  to[out=190,in=75] node[above] {$\gamma_\pi$}
                  (1,.2);
                  
    \foreach \p in {M} \draw node[dot,label=below:$\p$] at (\p) {};
  \end{tikzpicture}
  \caption[Ортогональная проекция кривой на касательную
  плоскость]{Кривая $\gamma$ на поверхности $\Phi$ и её проекция
    $\gamma_\pi$ на касательную в точке $M$ плоскость $\pi$}
  \label{fig:geodesic-curvature}
\end{figure}

Геодезическая кривизна кривой $\gamma = \gamma(x^1(s), x^2(s))$,
параметризованной естественно ($s$ — длина дуги) во всякой точке может
быть вычислена по формуле (см. \cite{pogorelov74}, \cite{rashevsky50})
\begin{equation}
  k_g = \sqrt{g_{11} g_{22} - g_{12}^2}\abs{\dot{x}^1 (\ddot{x}^2 +
    \krist{I}{J}{2} \dot{x}^I \dot{x}^J) - \dot{x}^2 (\ddot{x}^1 +
    \krist{I}{J}{1} \dot{x}^I \dot{x}^J)}
\end{equation}
где $\krist{I}{J}{K}$ — \neword{символы Кристоффеля второго рода},
применяемые при различных вычислениях параметров поверхности:
\begin{equation}\label{eq:krist-generic}
  \krist{I}{J}{K} = \frac{1}{2} g^{LK} \left (
    \frac{\partial{g_{IL}}}{\partial{x^J}} +
    \frac{\partial{g_{JL}}}{\partial{x^I}} -
    \frac{\partial{g_{IJ}}}{\partial{x^L}} \right )
\end{equation}

Вычисляя символы Кристоффеля по формуле \eqref{eq:krist-generic},
получим
\begin{equation}\label{eq:krist}
  \begin{aligned}
    \krist{1}{1}{1} &= 0 &\qquad \krist{1}{1}{2} &= 0 \\
    \krist{1}{2}{1} &= 0 &\qquad \krist{1}{2}{2} &= \frac{\ch{\rho}}{\sh{\rho}} \\
    \krist{2}{1}{1} &= 0 &\qquad \krist{2}{1}{2} &= \frac{\ch{\rho}}{\sh{\rho}} \\
    \krist{2}{2}{1} &= -\ch{\rho} \sh{\rho} &\qquad \krist{2}{2}{2} & = 0
  \end{aligned}
\end{equation}

Таким образом, окончательное выражение для нахождения геодезической
кривизны кривой с естественной параметризацией $(\rho(s),\,\phi(s))$ в
рассматриваемом пространстве имеет вид
\begin{equation}\label{eq:geodesic-curvature}
  k_g = \sh\rho \abs{\dot\rho(\ddot\phi +
    \krist{1}{2}{2}\dot\rho\dot\phi +
    \krist{2}{1}{2}\dot\phi\dot\rho) -
    \dot\phi(\ddot\rho + \krist{2}{2}{1}\dot\phi\dot\phi)}  
\end{equation}

\subsubsection{Геодезические линии}

\neword{Геодезической линией} на поверхности называется кривая,
геодезическая кривизна которой в каждой точке равна нулю.
Геодезические линии являются прямыми или обладают нормалями,
совпадающими с нормалями к рассматриваемой поверхности.

Дифференциальное уравнение геодезических линий в координатной форме
имеет вид (см. \cite{dubrovin98})
\begin{equation}
  \frac{d^2x^K}{ds^2} + \krist{I}{J}{K} \frac{dx^I}{ds}
  \frac{dx^J}{ds} = 0
\end{equation}

Кроме того, известно, что геодезические линии должны удовлетворять
соотношению \eqref{eq:lin-elt} определяемому линейным элементом.

В рассматриваемом случае получаем систему дифференциальных уравнений
\begin{subnumcases}{}
  \ddot{\rho} - \ch{\rho}\sh{\rho}\dot{\phi} = 0 \\
  \ddot{\phi} + 2\frac{\ch{\rho}}{\sh{\rho}}\dot{\rho} \dot{\phi} = 0
  & \label{eq:geo-des-second}\\
  (\dot{\rho})^2  = 1 - (\dot{\phi})^2 \sh^2{\rho} 
  & \label{eq:geo-des-third}
\end{subnumcases}

где точками обозначены производные соответствующих порядков по длине
дуги $s$, а \eqref{eq:geo-des-third} получено из \eqref{eq:lin-elt}.

Решим эту систему, получив уравнение геодезических линий.

Рассмотрим два случая.
\begin{rlist}
\item Пусть $\dot{\phi} = 0 \iff \phi = \hat{C}_1 = \const$. Получаем систему
  \begin{subnumcases}{\label{eq:geo-des2}}
    \ddot{\rho} = 0 \\
    \ddot{\phi} = 0 \\
    (\dot{\rho})^2  = 1 & \label{eq:geo-des2-third}
  \end{subnumcases}

  Из \eqref{eq:geo-des2-third}
  \begin{equation*}
    \begin{split}
      \dot{\rho} =& \pm 1\\
      \rho =& \pm s + \hat{C}_2
    \end{split}
  \end{equation*}
  
  Этот случай описывает прямые, проходящие через начало полярной
  системы координат с постоянным полярным углом $\phi$, при этом знак
  и постоянная константа $\hat{C}_2$ в выражении для $\rho$ не
  оказывают влияния на характер таких геодезических.
  
\item Пусть теперь $\dot{\phi} \neq 0$. Из \eqref{eq:geo-des-second} получим:
  \begin{equation*}
    \begin{split}
      \frac{\ddot{\phi}}{\dot{\phi}} =& -2 \frac{\ch \rho}{\sh \rho}\dot{\rho} \\
      \left( \ln{\dot{\phi}} \right )^{\prime} =& -2 \left( \ln (\sh \rho) \right)^{\prime} \\
      \ln{\dot{\phi}} =& -2 \ln (\sh \rho) + C \\
      \dot{\phi} =& \frac{C_1}{\sh^2\rho}\\
      (\dot{\phi})^2 =& \frac{C_1^2}{\sh^4\rho}
    \end{split}
  \end{equation*}
  
  Разделим \eqref{eq:geo-des-third} на полученное выражение для $(\dot{\phi})^2$:
  \begin{equation*}
    \begin{split}
      \left ( \frac{d\rho}{d\phi} \right )^2 =& \frac{\sh^4\rho}{C_1^2} - \sh^2\rho \\
      \frac{d\rho}{d\phi} =& \sqrt{\frac{\sh^4\rho}{C_1^2}-\sh^2\rho} =
      \pm \sh^2\rho \sqrt{\frac{1}{C_1^2}-\frac{1}{\sh^2\rho}} \\
      \frac{d\phi}{d\rho} =& \frac{1}{\pm \sh^2\rho \sqrt{\frac{1}{C_1^2}-\frac{1}{\sh^2\rho}}} \\
    \end{split}
  \end{equation*}
  
  Теперь можно определить характер зависимости $\phi(\rho)$:
  \begin{multline}\label{eq:geo-function}
    \phi(\rho) = \pm \int \frac{d\rho}{\sh^2\rho
      \sqrt{\frac{1}{C_1^2}-\frac{1}{\sh^2\rho}}} =
    \mp \int\frac{d(\cth\rho)}{\sqrt{\left(1 + \frac{1}{C_1^2} \right)-\cth^2 \rho}} = \\
    = \mp \arcsin\left(\frac{\cth\rho}{\pm\sqrt{1 + \frac{1}{C_1^2}}}\right) + C_2 =
    \mp \arcsin\left(\frac{\cth\rho}{\widetilde{C}_1}\right) + C_2
  \end{multline}
  где $\widetilde{C}_1 = \pm\sqrt{1+1/C_1^2}$, так что
  $\lvert\widetilde{C}_1\rvert>1$. При выводе \eqref{eq:geo-function}
  использовалось соотношение $\cth^2\rho = {\ch^2\rho}/{\sh^2\rho} =
  1+{1}/{\sh^2\rho}$.

\end{rlist}

Подробнее рассмотрим второе полученное семейство геодезических.

Запишем уравнение \eqref{eq:geo-function} в параметрическом виде:
\begin{equation}\label{eq:geo-function-par}
  \begin{cases}
    \rho(t) = t \\
    \phi(t) = \pm \arcsin\left(\frac{\cth t}{C_1}\right) + C_2
  \end{cases}
\end{equation}

Арксинус — нечётная функция, а выражение для $\phi(t)$ в этой системе
имеет знак $\pm$, поэтому константу $C_1$ без потери общности можно
считать строго положительной, поэтому $C_1 > 1$.

Определим значения $t$, при которых определены значения $\phi(t)$ в
\eqref{eq:geo-function-par}. Поскольку параметр $t$ равен полярному
радиусу, в дальнейшем будем рассматривать $t>0$, так что $\cth t > 0$.
Также отметим, что значение $C_2$ не оказывает влияния на область
определения.

Функция $\arcsin{\left(\frac{\cth{t}}{C_1}\right)}$ определена при
$\frac{\cth{t}}{C_1} \leq 1$. Поскольку \mbox{$\lim
  \limits_{t\to 0}{\cth{t}} = \infty$}, функция $\phi(t)$ в нуле не
определена для всякого конечного $C_1$, так что геодезические,
описываемые уравнением \eqref{eq:geo-function-par}, не проходят через
начало координат.

Кроме того, из свойств гиперболического котангенса
\begin{align*}
  \cth t &> 1 \\
  \lim_{t \to +\infty}{\cth{t}} &= 1 + 0
\end{align*}

\begin{figure}[thb]
  \centering
  \begin{tikzpicture}[scale=1.4,axis/.style={draw=gray,thin},samples=50,smooth]
    \draw[->,axis] (-0.2,0) -- (5.5,0) node[below left] {$t$};
    \draw[->,axis] (0,-0.2) -- (0,5.5) node[below left] {$f(t)$};

    \begin{scope}[domain=0.2:4]
      \draw[thick] plot[id=cth] function{cosh(x)/sinh(x)} node[above] {$\cth{t}$};
      \draw plot[id=cth2] function{cosh(x)/(2.2*sinh(x))} node[right] {$\frac{\cth{t}}{C_1}$};
    \end{scope}

    % cut out interval
    \coordinate[label=below left:$0$] (M) at (0,0) {};
    \coordinate[label=below:$t_0$] (N) at (0.49,0) {};
    \draw[very thick, densely dashed] (M)--(N);
    \draw[densely dotted] (N) -- ($ (N)+(0,1) $);
    
    % asymptote
    \draw[dashed] plot[id=one,domain=0:5] function{1};

    \node[left] at (0,1) {$1$};
  \end{tikzpicture}
  \caption[Гиперболический котангенс]{Функции $f(t) = \cth(t)$ и $f(t) = \frac{\cth(t)}{C_1},\,
    C_1>1,\,t>0$}
  \label{fig:coth}
\end{figure}


Очевидно,
\begin{align*}
  \frac{\cth t}{C_1} &> \frac{1}{C_1} \\
  \lim_{t \to +\infty}{\frac{\cth{t}}{C_1}} &= \frac{1}{C_1} + 0
\end{align*}

Отсюда следует, что в рассматриваемой функции $\phi(t)$ значение
константы $C_1$ необходимо больше единицы.

Значит, для всякого заданного $C_1$ функция $\phi(t)$ определена всюду
вне интервала $[0;t_0)$, где $t_0 = \arccth{C_1}$ (рисунок
\ref{fig:coth}). Действительно, поскольку гиперболический котангенс
убывает, при $t \geq t_0$ выполняется неравенство $\frac{\cth t}{C_1}
\leq \frac{C_1}{C_1} = 1$ , так что значение
$\arcsin{\frac{\cth t}{C_1}}$ определено.

Зафиксируем $C_2 = 0$ и рассмотрим ветвь \eqref{eq:geo-function-par} с
$\phi(t) = \arcsin\frac{\cth t}{C_1}$. При минимально возможном $t =
t_0 = \arccth{C_1}$ получаем $\rho = \arccth{C_1}$, $\phi =
\frac{\pi}{2}$. С ростом $t$ кривая удаляется от начала координат,
прижимаясь к асимптоте $\phi = \lim \limits_{t \to +\infty}
\arcsin{\left(\frac{\cth t}{C_1}\right)} = \arcsin{\frac{1}{C_1}}$.

Аналогично, другая ветвь \eqref{eq:geo-function-par} с $\phi(t) =
-\arcsin\frac{\cth{t}}{C_1}$ прижимается к прямой с $\phi =
-\arcsin{\frac{1}{C_1}}$.

Рассматривая дополнительно геодезические с тем же значением $C_1$ при
$C_2 = \pi$, вновь получим две ветви c асимптотами $\phi =
\pi-\arcsin\frac{1}{C_1} \sim -\arcsin\frac{1}{C_1}$ и $\phi =
\pi+\arcsin\frac{1}{C_1} \sim \arcsin\frac{1}{C_1}$, соответственно.

\begin{figure}[thb]
  \centering
  \begin{tikzpicture}[asymp/.style={dashed,draw=gray,thin},samples=50,smooth]
    \coordinate[label=left:$O$] (O) at (0,0) {};
    
    \begin{scope}[raw gnuplot]
      % asymptotes
      \begin{scope}[asymp]
        \begin{scope}[right]
          \draw plot[id=asymp1]
          function{set polar; set parametric; plot [-4:4] pi/6, t}
          node {$\pi/6$};
          \draw plot[id=asymp2]
          function{set polar; set parametric; plot [-4:4] pi/4, t}
          node {$\pi/4$};
          \draw plot[id=asymp3]
          function{set polar; set parametric; plot [-4:4] pi/3, t}
          node {$\pi/3$};
        \end{scope}
        \begin{scope}[left]
          \draw plot[id=_asymp1]
          function{set polar; set parametric; plot [-4:4] 5*pi/6, t}
          node {$5\pi/6$};
          \draw plot[id=_asymp2]
          function{set polar; set parametric; plot [-4:4] 3*pi/4, t}
          node {$3\pi/4$};
          \draw plot[id=_asymp3]
          function{set polar; set parametric; plot [-4:4] 2*pi/3, t}
          node {$2\pi/3$};
        \end{scope}
      \end{scope}
      \begin{scope}[thick]
        \draw plot[id=geo1] function{set polar; set parametric; plot [0.54930615:3] asin(cosh(t)/(2*sinh(t))),t;};
        \draw plot[id=_geo1] function{set polar; set parametric; plot [0.54930615:3] pi-asin(cosh(t)/(2*sinh(t))),t;};
        \draw plot[id=geo2] function{set polar; set parametric; plot [1.316958:3] asin(sqrt(3)*cosh(t)/(2*sinh(t))),t;};
        \draw plot[id=_geo2] function{set polar; set parametric; plot [1.316958:3] pi-asin(sqrt(3)*cosh(t)/(2*sinh(t))),t;};
        \draw plot[id=geo3] function{set polar; set parametric; plot [0.881374:3] asin(sqrt(2)*cosh(t)/(2*sinh(t))),t;};
        \draw plot[id=_geo3] function{set polar; set parametric; plot [0.881374:3] pi-asin(sqrt(2)*cosh(t)/(2*sinh(t))),t;};
        \draw plot[id=geo1] function{set polar; set parametric; plot [0.54930615:3] -asin(cosh(t)/(2*sinh(t))),t;};
        \draw plot[id=_geo1] function{set polar; set parametric; plot [0.54930615:3] -pi+asin(cosh(t)/(2*sinh(t))),t;};
        \draw plot[id=geo2] function{set polar; set parametric; plot [1.316958:3] -asin(sqrt(3)*cosh(t)/(2*sinh(t))),t;};
        \draw plot[id=_geo2] function{set polar; set parametric; plot [1.316958:3] -pi+asin(sqrt(3)*cosh(t)/(2*sinh(t))),t;};
        \draw plot[id=geo3] function{set polar; set parametric; plot [0.881374:3] -asin(sqrt(2)*cosh(t)/(2*sinh(t))),t;};
        \draw plot[id=_geo3] function{set polar; set parametric; plot [0.881374:3] -pi+asin(sqrt(2)*cosh(t)/(2*sinh(t))),t;};
      \end{scope}
    \end{scope}

    \node[dot] at (O) {};
  \end{tikzpicture}
  \caption{Геодезические линии с разными значениями параметра $C_1$}
  \label{fig:geodesics-hyperbolic}
\end{figure}


На рисунке \ref{fig:geodesics-hyperbolic} приведены три пары
геодезических линий для значений $C_1=2/\sqrt{3},\,2/\sqrt{2},\,2$.
При $C_2=0$ соответствующие асимптоты имеют полярные углы, равные
$\pi/3,\,\pi/4,\,\pi/6$, а при $C_2=\pi$ — углы
$2\pi/3,\,3\pi/4,\,5\pi/6$.

Видно, что с увеличением параметра $C_1$ увеличивается минимальное
расстояние до центра координат $\rho(t_0)$ и угол между асимптотами
пары геодезических линий для заданного $C_1$ и значений $C_2=0,\,\pi$.

\begin{figure}[!thb]
  \centering
  \begin{tikzpicture}[samples=50,smooth]
    \coordinate[label=left:$O$] (O) at (0,0) {};
    
    \begin{scope}[raw gnuplot]
      \foreach \p in {0,1,2,3,4,5,6,7,8,9,10,11}
               \draw plot function{set polar; set parametric; plot [0.54930615:2.5] asin(cosh(t)/(2*sinh(t)))+\p*pi/6,t;}
                     plot function{set polar; set parametric; plot [0.54930615:2.5] pi-asin(cosh(t)/(2*sinh(t)))+\p*pi/6,t;};
    \end{scope}

    \node[dot] at (O) {};
  \end{tikzpicture}
  \caption{Геодезические линии с разными значениями параметра $C_2$
    при \mbox{$C_1=2$}}
  \label{fig:geodesics-hyperbolic-2}
\end{figure}


Изменение параметра $C_2$ соответствует повороту выбранной
геодезической линии вокруг центра координат (рисунок
\ref{fig:geodesics-hyperbolic-2}).

Отметим, что при $C_1 \to +\infty$ минимальное допустимое $t \to 0$, а
всё семейство \eqref{eq:geo-function-par} вырождается в множество
прямых, проходящих через начало координат — то есть семейство
геодезических, описываемых уравнением \eqref{eq:geo-des2}.

\begin{figure}[thb]
  \centering
  \begin{tikzpicture}[samples=50,smooth,raw gnuplot,scale=2]
    \coordinate[] (O) at (0,0) {};
    
    % geodesics
    \foreach \f in {0,...,7}
    \foreach \r in {pi/4*\f}
    {\begin{scope}[thick]
        \draw plot function{set polar; set parametric; plot [0.54930615:3] asin(cosh(t)/(2*sinh(t)))+\r,t;};
        \draw plot function{set polar; set parametric; plot [0.54930615:3] pi-asin(cosh(t)/(2*sinh(t)))+\r,t;};
        \draw plot function{set polar; set parametric; plot [1.316958:3] asin(sqrt(3)*cosh(t)/(2*sinh(t)))+\r,t;};
        \draw plot function{set polar; set parametric; plot [1.316958:3] pi-asin(sqrt(3)*cosh(t)/(2*sinh(t)))+\r,t;};
        \draw plot function{set polar; set parametric; plot [0.881374:3] asin(sqrt(2)*cosh(t)/(2*sinh(t)))+\r,t;};
        \draw plot function{set polar; set parametric; plot [0.881374:3] pi-asin(sqrt(2)*cosh(t)/(2*sinh(t)))+\r,t;};

        \begin{scope}[densely dotted]
          \draw plot function{set polar; set parametric; plot [0:3.2] \r,t;};
          \draw plot function{set polar; set parametric; plot [0:3.2] \r+pi/6,t};
          \draw plot function{set polar; set parametric; plot [0:3.2] \r+pi/3,t};
        \end{scope}
      \end{scope}}

    \node[fill=white, anchor=east] at (O) {$O$};
    \node[dot] at (O) {};
  \end{tikzpicture}
  \caption{Геодезические линии на плоскости Лобачевского}
  \label{fig:geodesics-hyperbolic-all}
\end{figure}


Итак, мы определили, что геодезические линии на рассматриваемой
плоскости представляют собой прямые, проходящие через начало
координат, и всевозможные «гиперболы», симметричные относительно
начала координат (рисунок \ref{fig:geodesics-hyperbolic-all}).

\clearpage

\subsection{Угол между двумя кривыми}

Углом $\alpha$ между двумя кривыми $\gamma_1$ и $\gamma_2$ в точке их
пересечения $t_0$ называется угол между касательными векторами
скорости к кривым в данной точке. Пусть кривые заданы параметрически
как $l_1 = (\rho_1(t), \phi_1(t))$ и $l_2 = (\rho_2(t), \phi_2(t))$.
Тогда касательные вектора к ним определяются как $\vec{l_1}(t) =
(d\rho_1/dt,\,d\phi_1/dt)$ и $\vec{l_2}(t) =
(d\rho_2/dt,\,d{\phi_2}/dt)$, соответственно.

Тогда угол между
кривыми может быть определён по формуле (см. \cite{dubrovin98})
\begin{equation}\label{eq:curves-angle}
  \cos \alpha = \frac{\scalmult{\vec{l_1}(t_0), \vec{l_2}(t_0)}{}}{\abs{\vec{l_1}(t_0)} \abs{\vec{l_2}(t_0)}}
\end{equation}

Здесь $\abs{\vec{l}} = \sqrt{\scalmult{\vec{l},\vec{l}}{}}$.

\subsection{Гиперболический тангенс}

Приведём выражение, которым будем неоднократно пользоваться впоследствии:
\begin{equation}\label{eq:tanh}
  \frac{e^x-1}{e^x+1} =
  \frac{2(e^{x/2}-e^{-x/2})e^{x/2}}{2(e^{x/2}+e^{-x/2})e^{x/2}}
  = \frac{\sh(x/2)}{\ch(x/2)} = \th{\frac{x}{2}}
\end{equation}

\clearpage

\section{Спираль Архимеда}

По определению, в полярных координатах $(\rho, \phi)$ спираль Архимеда
$S_\alpha$ задаётся уравнением $\rho = \phi$, которое имеет
параметрический вид:
\begin{equation}\label{eq:arch-spiral}
  \begin{cases}
    x^1(t) = \rho(t) = t \\
    x^2(t) = \phi(t) = t
  \end{cases}
\end{equation}

\subsection{Длина дуги}

Длина дуги $\arc{AB}$ этой спирали, определяемой начальным и конечным значениеми
полярного угла $\phi_1$ и $\phi_2$ вычисляется по следующей формуле
(см. \cite{dubrovin98}):
\begin{equation}\label{eq:arch-length}
  \begin{split}
    l(S_\alpha)^{\phi_2}_{\phi_1} =
    \int \limits_{\arc{AB}}{ds} &=
    \int \limits_{\arc{AB}}{\sqrt{d\rho^2+\sh^2\rho\,d\phi^2}} =\\
    &=\int \limits^{\phi_2}_{\phi_1}{\sqrt{\left ( 1 + \sh^2\phi \right )}}\,d\phi =
    \int \limits^{\phi_2}_{\phi_1}\ch\phi\,d\phi =
    \sh{\phi_2} - \sh{\phi_1}
  \end{split}
\end{equation}

Таким образом, естественная параметризация спирали Архимеда имеет вид
\begin{equation}\label{eq:arch-natural-param}
  \begin{cases}
    \rho(s) = \arcsh{s} \\
    \phi(s) = \arcsh{s}
  \end{cases}
\end{equation}

\subsection{Геодезическая кривизна}

С учётом определения архимедовой спирали \eqref{eq:arch-spiral} и
результата \eqref{eq:arch-natural-param},
\begin{gather}
  \begin{align*}
    \dot{\rho} = \frac{1}{\sqrt{s^2+1}} && \ddot{\rho} = -\frac{s}{(s^2+1)^{\frac{3}{2}}}\\
    \dot{\phi} = \frac{1}{\sqrt{s^2+1}} && \ddot{\phi} = -\frac{s}{(s^2+1)^{\frac{3}{2}}}
  \end{align*}
\end{gather}

Символы Кристоффеля были вычислены в \eqref{eq:krist}. Подставим все
полученные выражения в \eqref{eq:geodesic-curvature}, учитывая, что
$x^1 \equiv \rho$, $x^2 \equiv \phi$:
\begin{equation}
  \begin{split}
    k_g =& \sh\rho \abs{\dot\rho(\ddot\phi +
      \krist{1}{2}{2}\dot\rho\dot\phi +
      \krist{2}{1}{2}\dot\phi\dot\rho) -
      \dot\phi(\ddot\rho + \krist{2}{2}{1}\dot\phi\dot\phi)} = \\
    =& \sh{\rho} \abs{\krist{1}{2}{2} \dot{\rho} \dot{\phi} +
      \krist{2}{1}{2} \dot{\phi} \dot{\rho} - \krist{2}{2}{1}
      \dot{\phi} \dot {\phi}}= \\
    =& \sh\rho \abs{\frac{\ch\rho}{\sh\rho} +
      \frac{\ch\rho}{\sh\rho} + \ch\rho \sh\rho} \frac{1}{s^2+1} = \\
    =& \frac{2\ch\rho + \ch\rho \sh^2\rho}{s^2+1}
    = \frac{(2+\sh^2\rho)\ch\rho}{s^2+1} = \\
    =& \frac{\ch\rho + \ch^3\rho}{s^2+1} =
    \frac{(\ch\rho+\ch^3\rho)}{\ch^2\rho} = \\
    =& \frac{1 + \ch^2\rho}{\ch\rho}
  \end{split}
\end{equation}

\clearpage
\subsection{Спираль Архимеда в моделях плоскости Лобачевского}

\subsubsection{Спираль Архимеда в модели Клейна}
\label{sec:arch-spiral-klein}

Рассмотрим модель Клейна (см. раздел \ref{sec:intro},
рисунок \ref{fig:klein-model}) в круге единичного радиуса. Обозначим
через $d_e$ обычную евклидову метрику, а через $d_k$ — метрику
\eqref{eq:projective-metric} в модели Клейна; $d_k$ определяется через
$d_e$:
\begin{equation}\label{eq:projective-metric-centered}
  d_k(O,P) = \frac{1}{2}\abs{\ln\left(\frac{d_e(X,O)}{d_e(X,P)}:\frac{d_e(Y,O)}{d_e(Y,P)}\right)}
\end{equation}

По аналогии рассмотрим системы полярных координат с полюсами в точке
$O$ (рисунок \ref{fig:klein-polar}): $(\rho_e, \phi_e)$, где
$\rho_e(P) \equiv d_e(O,P)$, а также $(\rho_k, \phi_e)$ с $\rho_k(P)
\equiv d_k(O, P)$. Отметим, что поскольку полюс выбран в центре круга,
$\phi_k \equiv \phi_e$ (в модели Клейна угол между прямыми,
проходящими через $O$, равен обычному евклидовому углу, см.
\cite{zaslavsky04}). В дальнейшем будем обозначать полярный угол как
$\phi$.

\begin{figure}[thb]
  \centering
  \begin{tikzpicture}[scale=2]
    \coordinate [label=left:$O$] (O) at (0, 0) {};
    \coordinate [label=above right:$X$] (X) at (40:1) {};
    \coordinate [label=below left:$Y$] (Y) at (220:1) {};
    \coordinate [label=above left:$P$] (P) at (40:0.7) {};

    % Klein disk itself
    \node [draw,thick,circle through=(X)] (circle) at (O) {};

    % chord
    \draw (X) -- (Y);
    \draw (O) -- (0:1);

    % coordinates
    \path (O)-- node[above,sloped,pos=0.5] {$\rho_e,\, \rho_k$} (P);
    \draw[very thin,draw=gray] (0:.7) arc (0:40:.7);
    \node at ($ (O)+(20:0.8) $) {$\phi$};

    % dot marks
    \foreach \p in {O, X, Y, P} \draw node[dot] at (\p) {};
  \end{tikzpicture}
  \caption{Системы полярных координат $(\rho_e,\phi_e)$ и
    $(\rho_k,\phi_k)$ в модели Клейна в круге единичного радиуса:
    $\rho_e \neq \rho_k,\, \phi_k=\phi_e=\phi$}
  \label{fig:klein-polar}
\end{figure}

С учётом определения $\rho_e,\,\rho_k$ и
\eqref{eq:projective-metric-centered},
\begin{multline}\label{eq:projective-metric-polar}
  \rho_k(P) =
  \frac{1}{2}\abs{\ln\left(\frac{\rho_e(X)}{\rho_e(X)-\rho_e(P)}:\frac{\rho_e(Y)}{\rho_e(Y)+\rho_e(P)}\right)}  = \\
  =\frac{1}{2}\abs{\ln\left(\frac{1}{1-\rho_e(P)}:\frac{1}{1+\rho_e(P)}\right)} =
  \frac{1}{2}\abs{\ln\left(\frac{1+\rho_e(P)}{1-\rho_e(P)}\right)} =\\
  = \frac{1}{2}\abs{\ln\left(1+\frac{2\rho_e(P)}{1-\rho_e(P)}\right)}
\end{multline}

Обозначим $\rho_k(P) = \rho_k$, $\rho_e(P) = \rho_e$. Из
\eqref{eq:projective-metric-polar} следует, что
\begin{align*}
  e^{2\rho_k} &= 1+\frac{2\rho_e}{1-\rho_e} \\
  e^{2\rho_k}-1 &= \frac{2}{1/\rho_e-1} \\
  \frac{2}{e^{2\rho_k}-1} &=1/\rho_e-1 \\
  \frac{e^{2\rho_k}+1}{e^{2\rho_k}-1} &= \frac{1}{\rho_e} \\
  \frac{e^{2\rho_k}-1}{e^{2\rho_k}+1} &= \rho_e \\
\end{align*}
и согласно \eqref{eq:tanh},
\begin{equation}\label{eq:klein-to-euclid}
  \th \rho_k = \rho_e
\end{equation}

Пусть в полярных координатах модели Клейна задана архимедова спираль
$\rho_k(\phi) = \phi$. Исходя из \eqref{eq:klein-to-euclid}, уравнение
такой спирали на евклидовой плоскости принимает вид
\begin{equation}\label{eq:arch-spiral-klein}
  \rho_e(\phi) = \th{\phi}
\end{equation}

\begin{figure}[thb]
  \centering
  \begin{tikzpicture}[scale=5]
    \coordinate [label=left:$O$] (O) at (0, 0) {};
    
    % Klein disk itself
    \draw[thick] (O) circle(1);

    \begin{scope}[decoration={
        markings,% switch on markings
        mark=% actually add a mark
        at position .07
        with \coordinate[] (P){};}]

      % Klein Archimedes spiral
      \draw[postaction={decorate}] (O) plot[parametric,samples=200,domain=0:12.56,id=exp,smooth]
                function{cos(t)*(1-2/(exp(2*t)+1)),sin(t)*(1-2/(exp(2*t)+1))};

    \end{scope}    
    
    \path[very thin,draw=gray] (O)-- node[sloped,above]{$\rho_k=\phi$} (P);
    
    % dot marks
    \foreach \p in {P,O} \draw node[dot] at (\p) {};
  \end{tikzpicture}
  \caption{Спираль Архимеда $\rho_k(\phi)=\phi$ в модели Клейна}
  \label{fig:arch-spiral-klein}
\end{figure}

На рисунке \ref{fig:arch-spiral-klein} видно, что кривая, заданная
уравнением \eqref{eq:arch-spiral-klein}, с увеличением $\phi$ всё
ближе прижимается к границе евклидовой единичной окружности
$\rho_e=1$, не достигая её (рисунок \ref{fig:arch-spiral-klein}).

Этот предсказуемый результат вновь поясняет смысл модели Клейна: с
приближением к границе круга меньшие в смысле евклидовой метрики
расстояния становятся всё большими в смысле метрики
\eqref{eq:projective-metric-polar}.

Аналогично, конечные в евклидовом пространстве хорды на
рисунке \ref{fig:klein-model} (см. страницу \pageref{fig:klein-model})
являются бесконечными прямыми в модели Клейна с другой метрикой.

\subsubsection{Спираль Архимеда в модели Пуанкаре в единичном круге}
\label{sec:arch-spiral-poincare}

Рассмотрим модель Пуанкаре в единичном круге (см. раздел
\ref{sec:what-is-up}, рисунок \ref{fig:poincare-disk-model};
рисунок \ref{fig:poincare-disk-model-2}) и две системы полярных
координат в ней — евклидову $(\rho_e, \phi)$ и систему $(\rho_p,
\phi)$, где $\rho_p$ — внутренняя метрика модели Пуанкаре. Вновь в
рамках рассматриваемой модели $\rho_e<1$, а угловые координаты обеих
систем совпадают. Изучим связь $\rho_p$ с $\rho_e$ с той же целью и
тем же подходом, что и в разделе \ref{sec:arch-spiral-klein}.

\begin{figure}[htb]
  \centering
  \begin{tikzpicture}[scale=2]
    \coordinate [label=below:$O$] (O) at (0, 0) {};
    \coordinate [label=above right:$Y$] (Y) at (55:1) {};
    \coordinate [label=below left:$X$] (X) at (160:1) {};
    \coordinate [label=right:$Z$] (Z) at (340:1) {};

    % disk model
    \node [draw,thick,circle through=(X)] (circle) at (O) {};

    % lines
    \draw (Y) to[out=235,in=-20] node[pos=0.5,label=above:$l$]{} (X)
       coordinate[pos=0.7,label=above:$B$](B){}
       coordinate[pos=0.3,label=above:$A$](A){};
    \draw (X)-- (Z) node[pos=0.3,label=below:$m$]{}
                    coordinate[pos=0.75,label=below:$P$](P){};
   
    %dot marks
    \foreach \p in {X, Y, Z, A, B, P, O} \draw node[dot] at (\p) {};
  \end{tikzpicture}
  \caption[Прямые и точки в модели Пуанкаре]{Прямые и точки в модели
    Пуанкаре: $l \parallel m$}
  \label{fig:poincare-disk-model-2}
\end{figure}

Для начала стоит рассмотреть связь модели Пуанкаре с моделью Клейна.

\begin{figure}[!hbt]
  \centering
  \begin{tikzpicture}[scale=1.85]
    \coordinate (O) at (0, 0) {};
    
    % Klein model points
    \coordinate (P) at (-1.42, .04) {};
    \coordinate (Q) at (-0.933,-0.48) {};

    % Projections onto sphere
    \coordinate (Ps) at (-1.42,-.85) {};
    \coordinate (Qs) at (-0.93,-1.3) {};

    % Sphere poles
    \coordinate [label=above:$N$] (N) at (0, 2) {};
    \coordinate (S) at (0, -2) {};

    % Disk control points
    \coordinate [] (L) at ($ (O)-(2,0) $) {};
    \coordinate [] (R) at ($ (O)+(2,0) $) {};

    % Projection lines
    \draw (P)--(Ps);
    \coordinate (Pp)
                at (intersection of Ps--N and O--P);                
    \draw (Ps)--(Pp);
    \draw (Q)--(Qs);
    \coordinate (Qp)
                at (intersection of Qs--N and O--Q);                
    \draw (Qs)--(Qp);

 
    % Fade lower hemisphere                 
    \path [fill=white, opacity=.5]
                (L) to[out=270,in=180] (S)
                    to[out=0,in=270] (R)
                    (O) ellipse(2 and 1);

     % Klein disk
    \draw [opacity=.3, pattern=north east lines]
               (O) ellipse(2 and 1);
    \draw [thick] (O) ellipse(2 and 1);

    % Chord
    \coordinate [dot] (X) at (-1.8,0.4358) {};
    \coordinate [dot] (Y) at (-0.464,-0.97) {};
    \draw[] (X)--(Y);
    \draw[] (X) to[out=-15,in=135,looseness=.9] (Pp)
                to [out=315,in=120,looseness=.7] (Qp)
                to[out=300,in=65,looseness=.8] (Y);
                    
    % Visible projection lines
    \draw (Pp)--(N);
    \draw (P)--(O);
    \draw (Qp)--(N);
    \draw (Q)--(O);

    % Sphere
    \draw[thick] (O) circle(2);

    % labels placed last
    \begin{scope}[inner sep=2pt]
      \node [label=below:$X$] at (X){};
      \node [label=below:$Y$] at (Y){};
      \node [label=right:$O$] at (O){};
      \node [label=left:$P$] at (P){};
      \node [label=left:$Q$] at (Q){};
      \node [label=above:$P^\prime$] at (Pp){};
      \node [inner sep=4pt,label=right:$Q^\prime$] at (Qp){};
      \node [label=below:$P_s$] at (Ps){};
      \node [label=below:$Q_s$] at (Qs){};
    \end{scope}
    
    % dot marks
    \foreach \p in {O,P,Pp,Q,Qp,N} \draw node[dot] at (\p) {};
    \node[dot,color=gray] at (Ps) {};
    \node[dot,color=gray] at (Qs) {};
  \end{tikzpicture}
  \caption{Связь моделей Клейна и Пуанкаре: $\xi(P) = P^\prime,\: \xi(Q)=Q^\prime$}
  \label{fig:klein-to-poincare}
\end{figure}

Известно, что сопоставление $\xi$ точки $P$ модели Клейна и
соответствующей ей точки $P^\prime$ модели Пуанкаре может быть
осуществлено следующим образом (рисунок \ref{fig:klein-to-poincare}):
на границе модели Клейна как на диаметре строится сфера единичного
радиуса, точка $P$ проецируется на нижнюю полусферу в точку $P_s$,
которая в свою очередь соединяется с северным полюсом сферы $N$.
Пересечение $NP_s$ с кругом модели Клейна даёт искомую точку
$P^\prime$ модели Пуанкаре в том же круге. Такая стереографическая
проекция также переводит прямые модели Клейна (хорды) в прямые модели
Пуанкаре (дуги окружностей, пересекающие границу круга под прямым
углом). Требуется, чтобы $\xi$ было изометрией, поэтому необходимо
\begin{equation}\label{eq:klein-to-poincare}
  d_k(P,Q) = d_p(P^\prime,Q^\prime)
\end{equation}

Из соображений проективной геометрии (см. \cite{prasolov04}) известно,
что при отображении $\xi$ метрика преобразуется следующим образом:
$2d_k(P,Q) = d_k(P^\prime, Q^\prime)$, где
$P^\prime=\xi(P),\,Q^\prime=\xi(Q)$. С учётом этого, а также
\eqref{eq:klein-to-poincare} и \eqref{eq:projective-metric-polar},
немедленно получаем
\begin{equation}\label{eq:poincare-metric-polar}
 \rho_p(P) = \abs{\ln\left(1+\frac{2\rho_e(P)}{1-\rho_e(P)}\right)}  
\end{equation}

Заметим, что метрику \eqref{eq:poincare-metric-polar} можно
получить и из соображений иного рода.

В модели Пуанкаре прямые, проходящие через центр круга $O$,
изображаются евклидовыми хордами, вдоль которых полярный угол $\phi$
очевидно, не меняется.

Применим \eqref{eq:poincare-lin-elt-polar} к задаче вычисления
полярного радиуса (в метрике $\rho_p$) точки $P$, лежащей на прямой
$OP$ модели Пуанкаре. Пусть ей соответствует евклидов полярный радиус
$\rho_{e_0}$.

С учётом $d\phi=0$, получим
\begin{subequations}
  \begin{equation*}
    \begin{split}
      ds^2&=4\frac{d\rho_e^2}{(1-\rho_e^2)^2} \\
      ds &= 2\frac{d\rho_e}{1-\rho_e^2} \\
    \end{split}
  \end{equation*}
  \begin{multline*}
    \rho_p = 2\int\limits_0^{\rho_{e_0}} \frac{d\rho_e}{1-\rho_e^2} =
     2 \frac{1}{2} \left . \left [ \ln(1+\rho_e) - \ln(1-\rho_e)
       \right ] \right\rvert_0^{\rho_{e_0}} = \\
     = \left . \ln{\left( \frac{1+\rho_e}{1-\rho_e} \right)} \right
     \rvert_0^{\rho_{e_0}} =
     \abs{ \ln{\left(\frac{1+\rho_{e_0}}{1-\rho_{e_0}} \right)}} =
     \abs{ \ln{\left(1+\frac{2\rho_{e_0}}{1-\rho_{e_0}} \right)}}
  \end{multline*}
\end{subequations}

Как видно, результат полностью совпадает с
\eqref{eq:poincare-metric-polar}.

Для модели Пуанкаре аналогичное \eqref{eq:klein-to-euclid} соотношение
принимает вид

\begin{equation}\label{eq:poincare-to-euclid}
  \th \frac{\rho_p}{2} = \rho_e
\end{equation}

Таким образом, спираль Архимеда $\rho_p = \phi$ в модели Пуанкаре на
евклидовой плоскости задаётся уравнением
\begin{equation}\label{eq:arch-spiral-poincare}
  \rho_e(\phi) = \th\frac{\phi}{2}
\end{equation}
Евклидов образ спирали приведён на рисунке \ref{fig:arch-spiral-poincare}.

\begin{figure}[htb]
  \centering
  \begin{tikzpicture}[scale=4]
    \coordinate [label=left:$O$] (O) at (0, 0) {};
    
    % Klein disk itself
    \draw[thick] (O) circle(1);

    \begin{scope}[decoration={
        markings,
        mark=at position .12
        with \coordinate[] (P){};}]

      % Poincare Archimedes spiral
      \draw[postaction={decorate}] (O) plot[parametric,samples=100,domain=0:12.56,id=exp,smooth]
                function{cos(t)*(1-2/(exp(t)+1)),sin(t)*(1-2/(exp(t)+1))};

    \end{scope}    
    
    \path[very thin,draw=gray] (O)-- node[sloped,above]{$\rho_p=\phi$} (P);
    
    % dot marks
    \foreach \p in {P,O} \draw node[dot] at (\p) {};
  \end{tikzpicture}
  \caption{Спираль Архимеда $\rho_p(\phi)=\phi$ в модели Пуанкаре}
  \label{fig:arch-spiral-poincare}
\end{figure}

\clearpage
\section{Спираль, пересекающая концентрические\\ окружности под постоянным углом}

\subsection{Уравнение спирали}

Рассматривается спираль, пересекающая концентрические окружности с
центром в начале координат под постоянным углом $\alpha$. Найдём
параметризацию искомой спирали $S(t) = (\rho(t), \phi(t))$.
Обозначим окружность радиуса $\rho_0$ как $O(t) =
(\bar{\rho}(t)=\rho_0,\,\bar{\phi}(t)=t)$

Из \eqref{eq:curves-angle}:
\begin{equation}
  \cos \alpha = \frac{\scalmult{\vec{O}, \vec{S}}{}}
                     {\abs{\vec{O}} \abs{\vec{S}}}
\end{equation}
С учётом этого, а также \eqref{eq:scalmult}:
\begin{align*}
  \cos \alpha &= \frac{d\bar{\rho} d\rho + \sh^2{\rho}\,d\bar{\phi}
    d\phi}{\sqrt{d\bar{\rho}^2 + \sh^2{\rho}\,d\bar{\phi}^2}
           \sqrt{d{\rho}^2 + \sh^2{\rho}\,d{\phi}^2}}\\
         \text{Поскольку $d\bar{\rho}=0$,}\\
  \cos \alpha &= \frac{\sh^2\rho\,d\bar{\phi} d\phi}
                      {\sqrt{\sh^2\rho\,d\bar{\phi}^2}
                       \sqrt{d{\rho}^2 + \sh^2\rho\,d{\phi}^2}}\\
  \cos \alpha &= \frac{\sh \rho\, d\phi}
                      {\sqrt{d\rho^2+\sh^2\rho\,d\phi^2}}\\
  \cos^2\alpha\,(d\rho^2+\sh^2\rho\,d\phi^2) &= \sh^2\rho\,d\phi^2 \\
  \cos^2\alpha\,d\rho^2 &= \sin^2\alpha \sh^2\rho\,d\phi^2\\
  \cos \alpha\,d\rho &= \pm \sin\alpha \sh\rho\, d\phi\\
  \ctg\alpha\,\frac{d \rho}{\sh\rho} &= d\phi\\
  \ctg\alpha\,\ln\left(\frac{e^\rho-1}{e^\rho+1}\right) &= \phi + C
\end{align*}

С учётом \eqref{eq:tanh}, для всякого выбранного значения угла
$\alpha$ параметризация спирали $S$ имеет вид
\begin{equation}\label{eq:con-spiral}
  \begin{cases}
    \rho(t) = t \\
    \phi(t) = \ctg\alpha \cdot \ln \left(\th\frac{t}{2}\right) + C
  \end{cases}
\end{equation}

Не ограничивая общности рассуждений, в дальнейшем константу
интегрирования $C$ для определённости будем полагать нулевой.

Известно, что указанным свойством (пересечение радиальных линий под
постоянным углом) \emph{в евклидовом пространстве} обладает
логарифмическая спираль (\cite{klein39}). Действительно, исходя из
найденного в разделе \ref{sec:arch-spiral-poincare} соотношения
\eqref{eq:poincare-to-euclid} между евклидовой метрикой $\rho_e$ и
метрикой $\rho_p$ модели Пуанкаре и того факта, что геометрия модели
Пуанкаре совпадает с рассматриваемой (см. \ref{sec:what-is-up}), можно
сказать, что если спираль $S(t)$ задана в модели Пуанкаре и для неё
$\rho_p = t$, то $\rho_e=\th(t/2)$. Поскольку $\phi_e \equiv \phi_p$,
то в евклидовом пространстве параметризация такой спирали имеет вид
\begin{equation*}
  \begin{cases}
    \rho(t) = \th\frac{t}{2} \\
    \phi(t) = \ctg\alpha \cdot \ln\left(\th\frac{t}{2}\right)
  \end{cases}
\end{equation*}
Выражая $\phi$ через $\rho$, получим
\begin{equation*}
  \begin{split}
    \phi(\rho(t)) &= \ctg\alpha \cdot \ln{\rho(t)} \\
    \tg\alpha \cdot \phi &= \ln\rho \\
    \rho(\phi) &= e^{\tg\alpha \cdot \phi}
  \end{split}
\end{equation*}
что соответствует известному уравнению логарифмической спирали.

\subsection{Длина дуги}

Вычислим длину $l=l(S)_{t_1}^{t_2}$ дуги $\arc{AB}$, которая
определяется начальным и конечным значениями параметра $t$, для
рассматриваемой спирали по формуле \eqref{eq:riemann-curve-length}:
\begin{equation*}
  l(S)_{t_1}^{t_2} = \int \limits_{\arc{AB}}{\sqrt{d\rho^2+\sh^2\rho\,d\phi^2}}
\end{equation*}
Сначала преобразуем подкоренное выражение с учётом
\eqref{eq:con-spiral}. Поскольку $d\phi = \frac{2 dt}{\tg\alpha
  \th(t/2) \ch^2(t/2)} = \frac{2 dt}{\tg\alpha \sh(t/2)
  \ch(t/2)}=\frac{dt}{\tg{\alpha}\sh{t}}$, то
\begin{equation*}  
  d\rho^2+\sh^2\rho\, d\phi^2 =
  dt^2 + \frac{\sh^2{t}\, dt}{\tg^2{\alpha}\sh^2{t}}
  =\left(1+\frac{1}{\tg^2{\alpha}}\right)dt^2
\end{equation*}
Поэтому формула для вычисления длины дуги принимает вид
\begin{multline}
  l(S)_{t_1}^{t_2} =
  \int \limits_{t_1}^{t_2}\sqrt{\left(1+\frac{1}{\tg^2{\alpha}}\right)dt^2} =
  \int \limits_{t_1}^{t_2}\sqrt{1+\frac{1}{\tg^2{\alpha}}}\,dt = \\
  = \int \limits_{t_1}^{t_2}\frac{dt}{\abs{\sin\alpha}} =
  \frac{(t_2 - t_1)}{\abs{\sin\alpha}} = \frac{(\rho_2 - \rho_1)}{\abs{\sin\alpha}}
\end{multline}

Таким образом при $0 < \alpha < \pi$, естественная параметризация
спирали $S$ представляется в виде
\begin{equation}
  \begin{cases}
    \rho(s) = s \sin\alpha\\
    \phi(s) = \ctg\alpha\cdot\ln\left(\th\frac{s \sin\alpha}{2}\right)
  \end{cases}
\end{equation}

\subsection{Геодезическая кривизна}
Производные от координат спирали по длине дуги $s$ таковы:
\begin{align*}
  \dot{\rho} &= \sin\alpha &\qquad \ddot{\rho} &= 0 \\
  \dot{\phi} &= \frac{\cos\alpha}{\sh\left(s \sin\alpha\right)} = \frac{\cos\alpha}{\sh\rho} &\qquad
  \ddot{\phi} &= -\frac{\sin{2\alpha} \cdot \ch\left(s \sin\alpha\right)}
                      {2 \sh^2\left(s \sin\alpha\right)} =
                      -\frac{\sin{2\alpha} \cdot \ch\rho}{2 \sh^2\rho}
\end{align*}

Подставим их, а также символы Кристоффеля \eqref{eq:krist} в
\eqref{eq:geodesic-curvature}:
\begin{equation*}\label{eq:con-geo-curvature}
  \begin{split}
    k_g = \sh\rho& \abs{\dot\rho(\ddot\phi +
      \krist{1}{2}{2}\dot\rho\dot\phi +
      \krist{2}{1}{2}\dot\phi\dot\rho) -
      \dot\phi(\ddot\rho + \krist{2}{2}{1}\dot\phi\dot\phi)} =\\
    =\sh\rho& \abs{\sin\alpha\left(-\frac{\sin{2\alpha} \cdot
          \ch\rho}{2\sh^2\rho} +
        \frac{\ch\rho}{\sh\rho}\left(2\cdot\sin\alpha\cdot\frac{\cos\alpha}{\sh\rho}\right)\right)
      - \frac{\cos\alpha}{\sh\rho}
      \left(\frac{\cos^2\alpha}{\sh^2\rho}\cdot\left(-\ch\rho
          \sh\rho\right)\right)} =\\
    =\sh\rho&
    \abs{\sin\alpha\left(\frac{\sin{2\alpha}\cdot\ch\rho}{2\sh^2\rho}\right)
      + \left(\frac{\cos^3\alpha\cdot\ch\rho}{\sh^2\rho}\right)} =\\
    =\cth\rho&\abs{\frac{\sin\alpha\cdot\sin{2\alpha}}{2}+\cos^3\alpha}
    = \cth\rho\abs{\cos\alpha\cdot\sin^2\alpha+\cos\alpha\,(1-\sin^2\alpha)} =\\
    = \cth\rho&\abs{\cos\alpha}    
  \end{split}
\end{equation*}

\clearpage
\appendix
\part{Информация о документе}

Данный документ был подготовлен с использованием \LaTeX{}. Иллюстрации
были созданы с помощью инструментов PGF и Ti$k$Z.

Автоматизация процесса сборки обеспечивалась утилитами
\texttt{GNU Make} и \texttt{texdepend}.

Представленная работа выполнена в рамках программы четвёртого семестра
обучения по специальности «Вычислительная математика и математическая
физика» в МГТУ им. Н. Э. Баумана.

Дата компиляции настоящего документа: \today.

\clearpage
\bibliographystyle{gost71s}
\bibliography{paper}
\end{document}
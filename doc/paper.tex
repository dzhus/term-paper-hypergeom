\documentclass{article}
\usepackage[utf8x]{inputenc}
\usepackage[english,russian]{babel}

\usepackage{amsmath,amssymb}

% provides enumerated cases environment
\usepackage{cases}

% provides \wideparen
\usepackage{yhmath}

% TikZ
\usepackage{tikz}
\usetikzlibrary{calc,through,positioning}
\tikzset{dot/.style={circle,fill=black,scale=0.5}}

\numberwithin{equation}{section}

\renewcommand{\epsilon}{\varepsilon}
\renewcommand{\phi}{\varphi}
\newcommand{\krist}[3]{\Gamma^{\phantom{#1 #2}#3}_{#1 #2}}
\newcommand{\neword}[1]{\textbf{#1}}
\newcommand{\scalmult}[2]{{\left \langle #1 \right \rangle}_{#2}}
\DeclareMathOperator{\arcsh}{arcsh}

\providecommand{\arc}[1]{\wideparen{#1}}
\providecommand{\abs}[1]{\left \lvert{#1}\right \rvert}

\begin{document}

\author{Дмитрий Джус}
\title{Курсовая работа по теме \\
  «Спирали на плоскости Лобачевского» \\
  (черновой вариант)}
\maketitle
\thispagestyle{empty}

\clearpage
\tableofcontents

\clearpage

\part{Введение}
\label{sec:intro}

\section{Исторический очерк}

В целях общего ознакомления с контекстом теории рассмотрим краткую
историю развития неевклидовой геометрии, в том числе геометрии
Лобачевского. Очерк подготовлен на основе \cite{milnor82} и
\cite{norden56}.

Геометрия Лобачевского (или гиперболическая геометрия) возникла в
попытках доказать или опровергнуть пятый постулат евклидовой
геометрии:

\begin{quote}
  Через любую точку, не лежащую на данной прямой, можно провести
  прямую, и притом только одну.
\end{quote}

Возможность опровержения этого начала занимала умы математиков и
философов на протяжении многих веков. Иммануил Кант с позиций своей
теории познания признавал \emph{априорный} характер геометрии и
утверждал, что основания геометрии имеют доопытное, очевидное
происхождение и черпаются из «чистого воззрения». С такой точки
зрения, лишь пятый постулат не обладал необходимым свойством
самоочевидности, и возможность его доказательства представлялась
равносильной возможности априорного обоснования всей геометрии.

Таких мыслей изначально придерживался и Карл Фридрих Гаусс.
Уверенность в возможности доказательства пятого постулата долго не
покидает его, и лишь в 1817 году он сомневается в этом, указывая на
вероятную неаприорность геометрии. В дальнейшие годы Гаусс значительно
изменяет своё видение теории параллельных, обнаруживая, что геометрия
без аксимы Евклида совершенно последовательна. При жизни Гаусс своих
работ по началам неевклидовой геометрии не публиковал.

В 1826 году Николай Иванович Лобачевский публикует своё сочинение
«О началах геометрии», содержащее непротиворечивую геометрическую
теорию, построенную с использованием не постулата Евклида о
параллельных, а предположения о том, что «угол параллелизма» наоборот
не является прямым и вовсе зависит от длины отрезка $a$. На рисунке
\ref{fig:lobachevsky} угол $\theta < \pi/2$ прямоугольного
треугольника $PQR$ является функцией $F(a)$ отрезка постоянной длины
$a$, увеличиваясь при $R \to \infty$.

\begin{figure}[thb]
  \centering
  \begin{tikzpicture}[scale=0.7]
    % triangle vertices
    \coordinate [label=below left:$P$] (P) at (0, 0);
    \coordinate [label=left:$Q$] (Q) at (0, 1.2cm);
    \coordinate [label=below:$R$] (R) at (6cm, 0) {};

    \draw (P) -- ($ (R) + .5*(2+rand, 0) $);

    % angle labels
    \node at ($ (P)!.5!(Q) $) [left] {$a$};
    \node at ($ (Q) + (-50:0.5cm) $){$\theta$};

    % hypotenuse and angle arc
    \begin{scope}
      \draw[clip] (P) -- node[left]{$a$} (Q) .. controls ($ (P)!.5!(R) $)
      and ($ (P)!.9!(R) $) .. (R);
      \draw (Q) circle(0.3cm);
    \end{scope}

    % dot marks
    \foreach \p in {P, Q, R} \draw node[dot] at (\p) {};
  \end{tikzpicture}
  \caption{Предположение Лобачевского: $\theta = F(a)$}
  \label{fig:lobachevsky}
\end{figure}

Указывая, что на предельной поверхности будет иметь место геометрия
Евклида, Лобачевский прямо переходит к доказательству
непротиворечивости тригонометрии. Впоследствии он также строит
дифференциальную и аналитическую геометрию своего пространства,
окончательно доказывая его непротиворечивость с одной стороны и
соответствие абсолютной геометрии с другой.

Работу по неевклидовой геометрии также опубликовал Янош Больаи в конце
1820-ых годов.

Дальнейшее развитие взглядов на геометрию Лобачевского было неотрывно
связано с дифференциальной геометрией.

В 1827 году Гаусс публикует свои работы по дифференциальной геометрии,
в которых даются определения таких понятий, как криволинейные
координаты, квадратичные формы поверхности, геодезическая кривизна.
Важным результатом Гаусса явилось выделение \emph{внутренней}
геометрии поверхности и доказательство инвариантности гауссовой
кривизны поверхности относительно преобразования изгибания. Известно,
что Гаусс также знал о псевдосфере — поверхности постоянной
отрицательной кривизны.

Ученик Гаусса Фердинанд Миндинг также рассматривал поверхности
постоянной кривизный, их движения по себе, заметив, что формулы
тригонометрии геодезических треугольников на поверхностях постоянной
отрицательной кривизны могут быть получены из тригонометрии на сфере
путём замены действительного коэффициента $\sqrt{k}$ на мнимый.

Опираясь на результаты Миндинга, Эудженио Бельтрами в 1868 публикует
своё сочинение об интерпретации неевклидовой геометрии. Согласно ему,
геометрия Лобачевского может быть осуществлена в евклидовом
пространстве как внутренняя геометрия поверхностей постоянной
отрицательной кривизны. Этот результат был воспринят как
доказательство непротиворечивости неевклидовой геометрии, хотя
Гильберт впоследствии указал на невозможность полной реализации
геометрии Лобачевского на таких поверхностях.

С этими открытиями и переводом работ Лобачевского повысился всеобщий
интерес к проблемам основания геометрии. Бернгард Риман рассмотрел
пространство $n$ измерений как многообразие элементов, в котором
каждый задаётся набором из $n$ независимых переменных, а геометрию
этого пространства можно определить квадратичной формой дифференциалов
независимых переменных (линейным элементом), которая задаёт квадрат
расстояния между элементами многообразия, что позволяет далее вводить
понятия длины, кратчайших или геодезических линий, угла, объёма. Риман
указал, что пространство, линейных элемент которого приводится к сумме
квадратов дифференциалов, является многомерным аналогом евклидового, а
также обобщил это понятие на пространства, допускающих движения по
себе с таким же числом степеней свободы, как и евклидово. Характер
этих пространств заключён в том, что их кривизна не зависит от
направления площадки.

Лобачевский положил в основу своей геометрии формули тригонометрии, а
Риман выполнил замысел Лобачевского ещё более законно, показав
возможность построения геометрии на чисто аналитической основе.
Впоследствии Бельтрами была обнаружена ещё более глубокая связь между
геометрией Лобачевского и системой Римана: трёхмерное пространство
Римана постоянной кривизны совпадает с пространством Лобачевского. Так
непротиворечивость неевклидовой геометрии была доказана строго.

Одновременно развивалась и проективная геометрия. Важной с точки
зрения обобщения геометрических понятий явилась мысль Юлиуса Плюккера
о том, что в качестве элементов геометрии могут быть рассмотрены не
только точки. В то же время, Феликс Клейн, опираясь на работы Артура
Кэли по проективной метрике, дал новую интерпретацию геометрии
Лобачевского с точки зрения геометрии проективной. Модель Клейна в
круге, приведённая на рисунке \ref{fig:klein-model}, является
полноценной моделью пространства Лобачевского, в которой метрика
задана следующим образом (см. \cite{prasolov04}):

\begin{equation}\label{eq:projective-metric}
  \rho(A, B) = \frac{R}{2} \abs{\ln{\left (
        \frac{X-A}{X-B}:\frac{Y-A}{Y-B} \right )}}
\end{equation}

В модели Клейна точки изображаются точками, прямые — хордами.
Бесконечное множество прямых, проходящих через точку $P$ параллельно
данной прямой $l$, заполняют часть круга между хордами $XM$ и $YN$.

\begin{figure}[thb]
  \centering
  \begin{tikzpicture}[]
    \coordinate [label=below left:$O$] (O) at (0, 0) {};
    \coordinate [label=above right:$X$] (X) at (60:3) {};
    \coordinate [label=below left:$Y$] (Y) at (200:3) {};
    \coordinate [label=above:$A$] (A) at ($ (Y)!.2!(X)$) {};
    \coordinate [label=above:$B$] (B) at ($ (X)!.3!(Y)$) {};
    \coordinate [label=above left:$P$] (P) at (285:1.5) {};

    % points somewhere outside the disk
    \coordinate (m) at ($ (X)!3!(P) $);
    \coordinate (n) at ($ (Y)!3!(P) $);

    % disk model
    \node [draw,thick,circle through=(X)] (circle) at (O) {};

    % chords
    \draw (X) --node[auto]{$l$} (Y);
    \coordinate [label=below:$M$] (M) at (intersection 1 of X--m and circle) {};
    \coordinate [label=right:$N$] (N) at (intersection 1 of Y--n and circle) {};
    \draw (X) -- (M);
    \draw (Y) -- (N);

    % a bundle of parallel lines
    \begin{scope}
      \clip (O) circle(3);
      \foreach \a in {5,15,25,35,45,55,65} \draw[ultra thin] ($ (P)+(180+\a:5) $) -- ($ (P)+(\a:5) $);
    \end{scope}

    %dot marks
    \foreach \p in {O, X, Y, A, B, P, M, N} \draw node[dot] at (\p) {};
  \end{tikzpicture}
  \caption{Модель Клейна в круге}
  \label{fig:klein-model}
\end{figure}

В конце XIX века Софус Ли создаёт теорию непрерывных групп
преобразований с многочисленными приложениями её к дифференциальным
уравнениями и геометрии. Понятие группы преобразований стало одним из
центральных в «Эрлангенской программе» Клейна, который в ней дал
обобщённоё понимание геометрии, определив её задачу следующим образом:

\begin{quote}
  Дано многообразие и в нём группа преобразований; нужно исследовать
  те свойства образов, принадлежащий многообразию, которые не
  изменяются от преобразований группы.
\end{quote}

Из такого определения напрямую вытекает существование различных
геометрий, которые различаются характером элементов и, что наиболее
важно, строением своей группы. Например, группа круговых
преобразований плоскости изоморфна группе движений пространства
Лобачевского, а значит, круговая геометрия плоскости даёт ещё одну
интерпретацию геометрии пространства Лобачевского.

Построение неевклидовой геометрии, впервые выполненное Лобачевским,
привело к интенсивному развитию геометрической теории, продолжающееся
и до сих пор по пути углубления и расширения сочетания групповой и
дифференциально-геометрической или более общей топологической точки
зрения.

\clearpage
\part{Практическая часть}

\section{Рассматриваемое пространство}

Рассматривается плоскость Лобачевского, первая квадратичная форма
задана в полярных координатах $(\rho, \phi)$ следующим образом:

\begin{equation}\label{eq:lin-elt}
  {ds}^2 = d\rho^2+\sh^2{\rho}\,d\phi^2
\end{equation}

\subsection{Линейный элемент}

Первая квадратичная форма (называемая линейным элементом) даёт
выражение для дифференциала дуги кривой.

В данном случае (форма невырожденная, положительно определённая на
всех точках) можно также говорить о том, что задана
\neword{риманова метрика} (см. \cite{fomenko00}).

Известно, что линейный элемент выражается как
\mbox{$ds^2=g_{ij}dx^idx^j$}, где коэффициенты $g_{ij}$ образуют
функциональную матрицу $G(x)$, соответствующую рассматриваемой
римановой метрике. В данном случае она имеет вид

\begin{equation*}
  \begin{pmatrix}
    1 && 0 \\
    0 && \sh^2{\rho}
  \end{pmatrix}
\end{equation*}

Задание линейного элемента на некоторой системе криволинейных
координат $(x^1, \dotsc , x^n)$ даёт возможность исследовать
\emph{внутреннюю} геометрию поверхности, в том числе, вычислять длины
дуг произвольных кривых:

\begin{equation}\label{eq:riemann-curve-length}
  l(\gamma)^a_b = \int \limits^a_b {\sqrt{\sum_{i,j}{g_{ij}
        \frac{dx^i}{dt} \frac{dx^j}{dt}}} dt} = \int \limits^a_b ds
\end{equation}

\subsection{Природа пространства}

Рассмотрим происхождение рассматриваемого пространства.

По аналогии с положительно определёнными римановыми метриками могут
быть введены \neword{индефинитные метрики}, соответствующая первая
квадратичная форма которых не обязательно обладает свойством
положительной определённости.

Примером индефинитных метрик служат \neword{псевдоевклидовы метрики}.
Для построения псевдоевклидовой метрики достаточно рассмотреть обычное
евклидово пространство $\mathbb{R}^n$ и задать в каждой его точке
билинейную форму с постоянными коэффициентами вида
\begin{equation*}
  \scalmult{\vec{\xi}, \vec{\eta}}{s} = -\sum_{i=1}^s {\xi^i \eta^i} +
\sum_{j=s+1}^n {\xi^j \eta^j}
\end{equation*}

Соответственно, длина дуги в этом \neword{псевдоевклидовом
  пространстве} $\mathbb{R}^n_s$ выражается по формуле
\begin{equation*}
  l(\gamma)^a_b = \int \limits^a_b {\sqrt{-\sum_{i=1}^s{\left
          (\frac{dx^i}{dt} \right )}^2 + \sum_{j=s+1}^n{\left
          (\frac{dx^j}{dt} \right )}^2}dt}
\end{equation*}

В $\mathbb{R}^n_s$ длина вектора может быть не только действительной,
но и нулевой или комплексной.

С введением возможности измерять расстояния в $\mathbb{R}^n_s$ может
быть рассмотрена сфера $S^{n-1}$ как множество точек, равноудалённых
от начала координат, при этом радиус не обязательно является
действительным.

Так, сфера нулевого радиуса описывается уравнением второго порядка
$-\sum_{i=1}^s\left (x^i \right)^2 + \sum_{i=s+1}^n\left(x^j \right)^2 = 0$, где
$x^1, \dotsc, x^n $ — декартовы координаты в $\mathbb{R}^n$, в котором
мы моделируем псевдоевклидово пространство $\mathbb{R}^n_s$.

Рассмотрим моделируемое в $\mathbb{R}^3$ пространство
$\mathbb{R}^3_1$; при помощи $x, y, z$ будем обозначать обычные
декартовы координаты в $\mathbb{R}^3$, так что $\scalmult{\vec{\xi},
  \vec{\xi}}{1} = -x^2 + y^2 + z^2$. Квадрат дифференциала длины дуги
в этом пространстве выражается как
\begin{equation}\label{eq:R^3_1-linear-element}
  ds^2 = -dx^2 + dy^2 + dz^2
\end{equation}

Рассмотрим сферу мнимого радиуса в $\mathbb{R}^3_1$. Это —
двуполостный гиперболоид, задаваемый уравнением $-\alpha^2 = -x^2 +
y^2 + z^2$. Изучим некоторые свойства его внутренней геометрии,
сначала спроектировав точки гиперболоида на плоскость следующим
образом.

Будем считать центром сферы $S^2_1=\left \{-\alpha^2 = -x^2 + y^2 +
  z^2\right \}$ точку $O=(0, 0, 0)$, а северным и южным полюсами — точки
$N=(-\alpha, 0, 0)$ и $S=(\alpha, 0, 0)$ соответственно.

\begin{figure}[!hb]
  \centering
  \begin{tikzpicture}[dot/.style={circle,fill=black,scale=0.5}]
    \coordinate [label=above left:$O$] (O) at (0, 0);
    \coordinate [label=right:$z$] (z) at (0, 3cm);
    \coordinate [label=below right:$y$] (y) at (-150:3cm);
    \coordinate [label=below:$x$] (x) at (5cm,0);
    
    \draw [->] (O) -- (x);
    \draw [->] (O) -- (y);
    \draw [->] (O) -- (z);
    
    \foreach \p in {O} \node[dot] at (\p){};
  \end{tikzpicture}
  \caption{Сфера $S^2_1$ мнимого радиуса, смоделированная в $\mathbb{R}^3$}
\end{figure}

Не ограничивая общности, для простоты рассмотрим ту часть сферы,
которая определена неравенством $x>0$.

Выберем плоскость $YOZ$, проходящую через центр сферы, в
качестве плоскости проецирования. Произвольную точку $P$ гиперболоида
соединим с северным полюсом $N$. Тогда образом точки $P$ при
стереографической проекции $\psi \colon S^2_1 \to \mathbb{R}^2$ будет
точка пересечения отрезка $PN$ с плоскостью $YOZ$. Видно, что образом
правой полости гиперболоида на плоскости будет внутренность диска $D^2$,
ограниченная неравенством $y^2 + z^2 < \alpha^2$, тогда как левая
полость проецируется во внешность этого диска, при этом граница диска
не принадлежит образу проекции.

Если точке гиперболоида $P = (x, y, z)$ при отображении $\psi \colon
{}_+S^2_1 \to D^2$ соответствует точка $f(P) = (u^1, u^2)$ на
плоскости проецирования, то их координаты связаны следующим образом
(см. \cite{fomenko00}):
\begin{equation}\label{eq:hyperboloid-to-plane}
  x = \alpha\frac{\abs{\vec{u}}^2 + \alpha^2}{\alpha^2 -
    \abs{\vec{u}}^2}, \qquad
  y = \frac{2 \alpha^2 u^1}{\alpha^2 - \abs{\vec{u}}^2}, \qquad
  z = \frac{2 \alpha^2 u^2}{\alpha^2 - \abs{\vec{u}}^2}
\end{equation}

Можно ввести в качестве точек внутренней геометрии рассматриваемой
сферы пары диаметрально противоположных точек $P$ и $-P$, а
прямыми объявить линии пересечения гиперболоида с всевозможными
плоскостями $ax+by+cz=0$, проходящими через центр сферы $O$.
Заметим, что параллельных прямых в данной геометрии в привычном для
нас понимании не существуют: прямые либо пересекаются в точке, либо
совпадают. Полученная геометрия называется \neword{эллиптической}.

При отображении $\psi$ прямые введённой геометрии на гиперболоиде
перейдут в дуги окружностей, пересекающие под прямым углом окружность
$y^2 + z^2 = \alpha^2$, что указывает на то, что геометрия, введённая
на сфере в $\mathbb{R}^3_1$ совпадает с геометрией, возникающей
в круге радиуса $\alpha$, взятого на евклидовой плоскости
$\mathbb{R}^2$, если в качестве точек этой геометрии взять точки
круга, а прямым назвать дуги окружностей, пересекающий границу круга
под прямым углом. Полученная геометрия и является геометрией
Лобачевского, а данная её модель называется \neword{моделью Пуанкаре в
  круге}. В ней выполнены все постулаты Евклида, кроме пятого. В
смысле возможности проведения параллельных прямых эта геометрия
противоположна эллиптической — через каждую точку в модели Пуанкаре
проходит бесчисленное число прямых, параллельных данной.
Стереографическая проекция $\phi$ сохраняет углы, так что они в модели
Пуанкаре изображаются обычными углами.

Теперь выполним следующее преобразование координат в $\mathbb{R}^3_1$.
Введём в плоскости $YOZ$ полярные координаты $(\rho, \phi)$, где
$\rho$ — угол с полярной осью $y$, и по аналогии со сферическими
координатами введём параметр $\theta$, равный углу между
радиус-вектором точки и проекцией этого радиус-вектора на $YOZ$.

После осуществления замены переменных
\begin{equation*}\label{eq:pseudospheric-coords}
  y = \alpha \sh{\theta} \cos{\phi}, \qquad
  z = \alpha \sh{\theta} \sin{\phi}, \qquad
  x = \alpha \ch{\theta}
\end{equation*}
уравнение сферы запишется в виде $\alpha = \text{const}$. Можно
получить вид \emph{римановой метрики на сфере} в координатах
$(u^1, u^2)$ модели Пуанкаре, подставляя формулы
\eqref{eq:hyperboloid-to-plane} в \eqref{eq:R^3_1-linear-element}:
\begin{equation*}
  -(dx(u^1, u^2))^2 + (dy(u^1, u^2))^2 + (dz(u^1, u^2))^2 =
  4\alpha^4\frac{(du^1)^2 + (du^2)^2}
                {\left (\alpha^2 - (u^1)^2-(u^2)^2 \right )^2}
\end{equation*}

Если в модели Пуанкаре при $\alpha=1$ теперь ввести полярные
координаты $(\rho, \phi)$, линейный элемент представится в виде
\begin{equation*}
  ds^2 = 4\frac{d\rho^2 + \rho^2 d\phi^2}{\left (1 - \rho^2 \right )^2}
\end{equation*}

Наконец, при записи в сферических координатах $(\theta, \phi)$
(что достигается выполнением преобразованием координат $\rho =
\cth(\theta/2)$, $\phi=\phi$) метрика приобретёт вид
\begin{equation}\label{eq:S^2_1-linear-element}
  ds^2 = d\theta^2 + \sh^2{\theta}d\phi^2
\end{equation}

Таким образом, на сфере $S^2_1$ мнимого радиуса объемлющим
псевдоевклидовым пространством $\mathbb{R}^3_1$ посредством
стереографического отображения $\psi$ \neword{индуцируется} риманова
метрика \eqref{eq:S^2_1-linear-element}. Отметим, что объемлющее
пространство обладает, наоборот, индефинитной метрикой
\eqref{eq:R^3_1-linear-element}.

Заметим, наконец, что выражение \eqref{eq:S^2_1-linear-element} для
линейного элемента \emph{совпадает} с предложенным к рассмотрению
\eqref{eq:lin-elt} с точностью до переобозначения координат $\theta =
\rho$.

Этот вывод, однако, не обладает свойством наглядности: в метрике
\eqref{eq:lin-elt} координата $\rho$ имеет смысл длины, в то время как
соответствующая ей в \eqref{eq:S^2_1-linear-element} величина $\theta$
— угловая.

\subsection{Геодезическая кривизна}

\neword{Геодезической кривизной} кривой $\gamma$ на поверхности $\Phi$
в некоторой точке $P$ называют кривизна ортогональной проекции $\gamma$ на
касательную плоскость к $\Phi$, проведённую в точке $\gamma$.

Геодезическая кривизна кривой $\gamma = \gamma(x^1(s), x^2(s))$,
параметризованной естественно ($s$ — длина дуги) во всякой точке может
быть вычислена по формуле (см. \cite{pogorelov74}, \cite{rashevsky50})
\begin{equation}\label{eq:geodesic-curvature}
  k_g = \sqrt{g_{11} g_{22} - g_{12}^2}\abs{\dot{x}^1 (\ddot{x}^2 +
    \krist{I}{J}{2} \dot{x}^I \dot{x}^J) - \dot{x}^2 (\ddot{x}^1 +
    \krist{I}{J}{1} \dot{x}^I \dot{x}^J)}
\end{equation}
где $\krist{I}{J}{K}$ — \neword{символы Кристоффеля второго рода},
применяемые при различные вычислениях параметров поверхности:
\begin{equation}\label{eq:krist-generic}
  \krist{I}{J}{K} = \frac{1}{2} g^{LK} \left (
    \frac{\partial{g_{IL}}}{\partial{x^J}} +
    \frac{\partial{g_{JL}}}{\partial{x^I}} -
    \frac{\partial{g_{IJ}}}{\partial{x^L}} \right )
\end{equation}

Вычисляя символы Кристоффеля по формуле \eqref{eq:krist-generic},
получим
\begin{equation}\label{eq:krist}
  \begin{aligned}
    \krist{1}{1}{1} &= 0 &\qquad \krist{1}{1}{2} &= 0 \\
    \krist{1}{2}{1} &= 0 &\qquad \krist{1}{2}{2} &= \frac{\ch{\rho}}{\sh{\rho}} \\
    \krist{2}{1}{1} &= 0 &\qquad \krist{2}{1}{2} &= \frac{\ch{\rho}}{\sh{\rho}} \\
    \krist{2}{2}{1} &= -\ch{\rho} \sh{\rho} &\qquad \krist{2}{2}{2} & = 0
  \end{aligned}
\end{equation}
  
\subsubsection{Геодезические линии}

\neword{Геодезической линией} на поверхности называется кривая,
геодезическая кривизна которой в каждой точке равна нулю.
Геодезические линии являются прямыми или обладают нормалями,
совпадающими с нормалями к рассматриваемой поверхности.

Дифференциальное уравнение геодезических линий в координатной форме
имеет вид
\begin{equation}
  \frac{d^2x^K}{ds^2} + \krist{I}{J}{K} \frac{dx^I}{ds}
  \frac{dx^J}{ds} = 0
\end{equation}

Кроме того, известно, что геодезические линии должны удовлетворять
соотношению \eqref{eq:lin-elt} определяемому линейным элементом.

В рассматриваемом случае получаем систему дифференциальных уравнений
\begin{subnumcases}{}
  \ddot{\rho} - \ch{\rho}\sh{\rho}\dot{\phi} = 0 \\
  \ddot{\phi} + 2\frac{\ch{\rho}}{\sh{\rho}}\dot{\rho} \dot{\phi} = 0
  & \label{eq:geo-des-second}\\
  (\dot{\rho})^2  = 1 - (\dot{\phi})^2 \sh^2{\rho} 
  & \label{eq:geo-des-third}
\end{subnumcases}

где точками обозначены производные соответствующих порядков по длине
дуги $s$, а \eqref{eq:geo-des-third} получено из \eqref{eq:lin-elt}.

Решим эту систему, получив уравнение геодезических линий.

Из \eqref{eq:geo-des-second} получим:
\begin{equation*}
  \begin{split}
    \frac{\ddot{\phi}}{\dot{\phi}} =& -2 \frac{\ch \rho}{\sh \rho}\dot{\rho} \\
    \left( \ln{\dot{\phi}} \right )^{\prime} =& -2 \left( \ln (\sh \rho) \right)^{\prime} \\
    \ln{\dot{\phi}} =& -2 \ln (\sh \rho) + C \\
    \dot{\phi} =& \frac{C_1}{\sh^2\rho}\\
    (\dot{\phi})^2 =& \frac{C_1^2}{\sh^4\rho}
  \end{split}
\end{equation*}

Разделим \eqref{eq:geo-des-third} на полученное выражение для $(\dot{\phi})^2$:
\begin{equation*}
  \begin{split}
    \left ( \frac{d\rho}{d\phi} \right )^2 =& \frac{\sh^4\rho}{C_1^2} + \sh^2\rho \\
    \frac{d\rho}{d\phi} =& \sqrt{\frac{\sh^4\rho}{C_1^2}-\sh^2\rho} =
    \sh^2\rho \sqrt{\frac{1}{C_1^2}-\frac{1}{\sh^2\rho}} \\
    \frac{d\phi}{d\rho} =& \frac{1}{\sh^2\rho \sqrt{\frac{1}{C_1^2}-\frac{1}{\sh^2\rho}}} \\
  \end{split}
\end{equation*}

Теперь можно определить характер зависимости $\phi(\rho)$:
\begin{multline}\label{eq:geo-function}
  \phi(\rho) = \int \frac{d\rho}{\sh^2\rho
    \sqrt{\frac{1}{C_1^2}-\frac{1}{\sh^2\rho}}} =
  -\int\frac{d(\cth\rho)}{\sqrt{\left(1 + \frac{1}{C_1^2} \right)-\cth^2 \rho}} = \\
  = -\arcsin\left(\frac{\cth\rho}{\pm\sqrt{1 + \frac{1}{C_1^2}}}\right) + C_2 =
  -\arcsin\left(\frac{\cth\rho}{\widetilde{C}_1}\right) + C_2
\end{multline}
где $\widetilde{C}_1 = \pm\sqrt{1+1/C_1^2}$, так что
$\lvert\widetilde{C}_1\rvert>1$. При выводе \eqref{eq:geo-function}
использовалось соотношение $\cth^2\rho = {\ch^2\rho}/{\sh^2\rho} =
1+{1}/{\sh^2\rho}$.

Запишем уравнение геодезических линий в параметрическом виде:
\begin{equation}
  \begin{cases}
    \rho(t) = t \\
    \phi(t) = -\arcsin\frac{\cth t}{\widetilde{C}_1} + C_2
  \end{cases}
\end{equation}

\clearpage

\section{Спираль Архимеда}

По определению, в полярных координатах $(\rho, \phi)$ спираль Архимеда
$\gamma_\alpha$ задаётся уравнением $\rho = \phi$, которое имеет
параметрический вид:
\begin{equation}\label{eq:arch-spiral}
  \begin{cases}
    x^1(t) = \rho(t) = t \\
    x^2(t) = \phi(t) = t
  \end{cases}
\end{equation}

\subsection{Длина дуги}

Длина дуги $\arc{AB}$ этой спирали, определяемой начальным и конечным значениеми
полярного угла $\phi_1$ и $\phi_2$ вычисляется по следующей формуле
(см. \cite{fomenko00}):
\begin{equation}\label{eq:arch-length}
  \begin{split}
    l(\gamma_\alpha)^{\phi_2}_{\phi_1} =
    \int \limits_{\arc{AB}}{ds} &=
    \int \limits_{\arc{AB}}{\sqrt{d\rho^2+\sh^2\rho\,d\phi^2}} =\\
    &=\int \limits^{\phi_2}_{\phi_1}{\sqrt{\left ( 1 + \sh^2\phi \right )}}\,d\phi =
    \int \limits^{\phi_2}_{\phi_1}\ch\phi\,d\phi =
    \sh{\phi_2} - \sh{\phi_1}
  \end{split}
\end{equation}

Таким образом, естественная параметризация спирали Архимеда имеет вид
\begin{equation}\label{eq:arch-natural-param}
  \begin{cases}
    \rho(s) = \arcsh{s} \\
    \phi(s) = \arcsh{s}
  \end{cases}
\end{equation}

\subsection{Геодезическая кривизна}

Найдём всё необходимое для вычисления выражения
\eqref{eq:geodesic-curvature}.

\begin{equation*}
  g^{IJ} = g_{IJ}^{-1} = \begin{pmatrix} 1 && 0 \\ 0 && 1/\sh^2{\rho} \end{pmatrix}
\end{equation*}

Кроме того, с учётом определения архимедовой спирали
\eqref{eq:arch-spiral} и результата \eqref{eq:arch-natural-param},
\begin{gather}
  \begin{align*}
    \dot{\rho} = \frac{1}{\sqrt{s^2+1}} && \ddot{\rho} = -\frac{s}{(s^2+1)^{\frac{3}{2}}}\\
    \dot{\phi} = \frac{1}{\sqrt{s^2+1}} && \ddot{\phi} = -\frac{s}{(s^2+1)^{\frac{3}{2}}}
  \end{align*}
\end{gather}

Символы Кристоффеля были вычислены в \eqref{eq:krist}. Подставим все
полученные выражения в \eqref{eq:geodesic-curvature}:
\begin{equation}\label{eq:arch-geo-curvature}
  \begin{split}
    k_g =& \sh{\rho} \abs{\krist{1}{2}{2} \dot{\rho} \dot{\phi} +
      \krist{2}{1}{2} \dot{\phi} \dot{\rho} - \krist{2}{2}{1} \dot{\phi}
      \dot {\phi}}= \\
    =& \sh\rho \abs{\frac{\ch\rho}{\sh\rho} +
      \frac{\ch\rho}{\sh\rho} + \ch\rho \sh\rho} \frac{1}{s^2+1} = \\
    =& \frac{2\ch\rho + \ch\rho \sh^2\rho}{s^2+1}
    = \frac{(2+\sh^2\rho)\ch\rho}{s^2+1} = \\
    =& \frac{\ch\rho + \ch^3\rho}{s^2+1} =
    \frac{(\ch\rho+\ch^3\rho)}{\ch^2\rho} = \\
    =& \frac{1 + \ch^2\rho}{\ch\rho}
  \end{split}
\end{equation}

В естественной параметризации геодезическая кривизна архимедовой
спирали при $\rho \geqslant 0$ вычисляется как $\frac{\sqrt{1+s^2} +
  (1+s^2)^{\frac{3}{2}}}{s^2+1}$, что следует из
\eqref{eq:arch-geo-curvature} и соотношения $\ch{\arcsh{s}} =
\pm\sqrt{1+s^2}$.

\clearpage
\bibliographystyle{gost780s}
\bibliography{paper}

\end{document}
\begin{figure}[thb]
  \centering
  \begin{tikzpicture}[samples=50,smooth,raw gnuplot]
    \coordinate[label=left:$O$] (O) at (0,0) {};
    
    % geodesics
    \foreach \r in {0, pi/2, pi, 3*pi/2}
    {\begin{scope}
        \draw plot function{set polar; set parametric; plot [0.54930615:3] asin(cosh(t)/(2*sinh(t)))+\r,t;};
        \draw plot function{set polar; set parametric; plot [0.54930615:3] pi-asin(cosh(t)/(2*sinh(t)))+\r,t;};
        \draw plot function{set polar; set parametric; plot [1.316958:3] asin(sqrt(3)*cosh(t)/(2*sinh(t)))+\r,t;};
        \draw plot function{set polar; set parametric; plot [1.316958:3] pi-asin(sqrt(3)*cosh(t)/(2*sinh(t)))+\r,t;};
        \draw plot function{set polar; set parametric; plot [0.881374:3] asin(sqrt(2)*cosh(t)/(2*sinh(t)))+\r,t;};
        \draw plot function{set polar; set parametric; plot [0.881374:3] pi-asin(sqrt(2)*cosh(t)/(2*sinh(t)))+\r,t;};
        \draw plot function{set polar; set parametric; plot [0:3.1] \r,t;};
        \draw plot function{set polar; set parametric; plot [0:3.1] \r+pi/6,t};
        \draw plot function{set polar; set parametric; plot [0:3.1] \r+pi/3,t};
      \end{scope}}
    
    \node[dot] at (O) {};
  \end{tikzpicture}
  \caption{Геодезические линии на плоскости Лобачевского}
  \label{fig:geodesics-hyperbolic-all}
\end{figure}

\begin{figure}[thb]
  \centering
  \begin{tikzpicture}[scale=1]
    \coordinate [label=left:$O$] (O) at (0, 0) {};
    \coordinate [label=above right:$X$] (X) at (40:3) {};
    \coordinate [label=below left:$Y$] (Y) at (220:3) {};
    \coordinate [label=above left:$P$] (P) at (40:2) {};

    % Klein disk itself
    \node [draw,thick,circle through=(X)] (circle) at (O) {};

    % chord
    \draw (X) -- (Y);
    \draw (O) -- (0:3);

    % coordinates
    \path (O)-- node[above,sloped,pos=0.5] {$\rho_e,\, \rho_k$} (P);
    \draw[very thin,draw=gray] (0:2) arc (0:40:2);
    \node at ($ (O)+(20:2.2) $) {$\phi$};

    % dot marks
    \foreach \p in {O, X, Y, P} \draw node[dot] at (\p) {};
  \end{tikzpicture}
  \caption{Системы полярных координат $(\rho_e,\phi_e)$ и
    $(\rho_k,\phi_k)$ в модели Клейна в круге единичного радиуса:
    $\rho_e \neq \rho_k,\, \phi_k=\phi_e=\phi$}
  \label{fig:klein-polar}
\end{figure}